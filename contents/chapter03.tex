% !TeX root = ../main.tex

\chapter{系統設計}

本章節將會介紹我們設計的系統,並且說明其架構、資料結構、威脅模型以及商業流程。

\section{系統的重要進步}

為了實現用戶對其身份的完全自主權, 本研究提出了許多創新的機制, 希望解決既有的技術困境。

\subsection{別名優先機制}

在過去的身份驗證系統中,會要求使用者提供一個唯一的帳號作為登入的識別。然而,這樣的設計無形中讓用戶對外交出了個人名稱的自主權,連名稱都無法自主的系統, 顯然難以達成本研究的設計目標。因此,我們提出讓用戶在註冊身份時優先使用自己喜歡的別名, 接著透過 UUID 的機制產生唯一識別號, 日常的操作中優先讓用戶使用別名作為帳號, 只有當別名因為重複而難以識別時會要求用戶使用和 UUID 關聯的機制做識別。當然,何時該使用別名,何時該使用 UUID ,這顯然是個複雜的問題,因此我們提出了"基於用戶時空的分析方法"試圖系統性的進行分析。

\subsection{基於用戶時空的分析方法}

提到物自體的概念...... 這樣的機制可以讓用戶在不同的情境下使用不同的身份,進而保護用戶的隱私。

\subsection{基於多因素驗證的可信度測量}

提到物自體的概念......

\subsection{基於危險程度的驗證機制}

人為自然立法的概念...... 這樣的機制可以讓用戶在不同的情境下使用不同的身份,進而保護用戶的隱私。

\subsection{極致多因素認證}

人為自然立法的概念......

\subsection{半匿名機制}

AID 可以拒絕被綁定到真實的人, 因為驗證核心自主掌握

AID 無法拒絕曾給出去的簽章, 因為簽章被他人自主掌握

\subsection{基於區塊鏈的自主轉址}

去中心化不代表無中心,....

\section{系統架構}
\subsection{整體說明}
\subsection{AID Server}
\subsection{Wallet}
\subsection{Consensus Core}

\section{資料結構}
\subsection{整體說明}
\subsection{AID Server}
\subsection{Wallet}
\subsection{Consensus Core}

\section{AID 的威脅模型}

\section{本章小結}

系統如何解決所有缺點,並且保留所有優點。