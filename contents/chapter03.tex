% !TeX root = ../main.tex

\chapter{系統設計}
本章節將會介紹我們設計的系統,並且說明其架構、資料結構以及威脅模型。
\section{系統的重要進步}
為實現用戶對其身份的完全自主權,本研究提出了多項創新機制,旨在解決現有的技術困境。
\subsection{別名優先機制}
傳統的身份驗證系統要求使用者提供唯一帳號作為登入識別。然而,這種設計無形中剝奪了用戶對個人名稱的自主權。一個連名稱都無法自主的系統,顯然難以達成本研究的設計目標。因此,我們提出讓用戶在註冊身份時優先使用自己喜歡的別名,再透過 UUID 機制產生唯一識別號。在日常操作中,系統優先讓用戶使用別名作為帳號,只有當別名因重複而難以識別時,才會要求用戶使用與 UUID 關聯的機制進行識別。當然,何時該使用別名,何時該使用 UUID,這是個複雜的問題。為此,我們提出了「基於用戶時空的分析方法」,試圖系統性地進行分析。
\subsection{基於用戶時空的分析方法}
每個身份本質上可視為一個隨時間變化的動態向量空間,其維度可謂無窮。每次對用戶的身份驗證都可視為取得特定時間僅含部分維度的向量。這方法的核心是讓用戶自主決定提供哪些維度,並在每次與系統互動時攜帶這些資訊。基於此概念,系統可藉由比對當次傳入的向量和近期暫存在快取中的所有向量來推測用戶的真實身份。

舉例來說:用戶在某次登入時提供了自己的別名、性別、所在地區等資訊,而在下次登入時僅提供了性別、所在地區等資訊,系統可以通過比對這兩次向量來推測用戶的別名。這種方法使用戶在不提供完整資訊的情況下,仍能通過部分資訊完成身份驗證。

然而,這種方法也帶來了一些問題。例如,用戶提供的維度多寡會影響系統的準確性,甚至用戶提供的維度是否包含可變資訊會影響系統的安全性。因此,我們提出維度的選擇甚至各個維度的權重應由用戶自主決定,而系統僅提供推薦機制。這樣可讓用戶在不同情境下使用不同方法,以滿足其需求。

總的來說,當僅有一個用戶被比較出來時視為可識別,有零個或多個用戶被比較出來時視為不可識別。這種機制受比較方法影響,因此我們提出了「基於危險程度的驗證機制」,希望能在不同情境下選擇適合的比較方法。
\subsection{基於危險程度的驗證機制}
不同的比較方法實質上是不同的嚴謹程度。在嚴謹程度較高的情況下,用戶可能需要更多維度符合,或需要在更接近的時間點內提供的紀錄才算數。這可能導致用戶難以找到識別對象,進而要求補充更多資訊再次驗證。相反,在嚴謹程度較低的情況下,用戶可能僅用少量維度的資訊配合較遠時間點的紀錄來完成驗證,這可能導致用戶被誤認為其他用戶,威脅系統安全。

考慮用戶行為的危險程度,我們設計出以下規則:
\begin{enumerate}
  \item 當用戶的行為被視為危險時,系統應提高驗證的嚴謹程度。
  \item 當用戶的行為被視為安全時,系統應降低驗證的嚴謹程度。
\end{enumerate}

接著,我們將嚴謹程度拆分為時間和空間兩個軸,使用笛卡爾座標系統表示,以細分出更複雜的情境:
\begin{enumerate}
  \item 當用戶行為超危險時,比較標準應為時間非常近或多個維度符合。
  \item 當用戶行為危險時,比較標準應為時間相對近或維度相對符合。
  \item 當用戶行為安全時,比較標準應為時間相對遠或維度相對不符合。
  \item 當用戶行為超安全時,比較標準應為時間非常遠或僅幾個維度符合。
\end{enumerate}

最後,用戶行為的危險程度在我們的系統中是由用戶自主決定的,系統僅提供推薦機制。我們希望藉此滿足所有用戶在各種情境下的需求。例如,對於個人銀行帳戶,可能所有行為都視為最高危險,因此需要最高的驗證嚴謹程度。而對於個人社群帳戶,讀取文章的行為可能視為不危險,發布文章則視為危險,因此需要不同的驗證嚴謹程度。我們建議用戶將危險的概念定義為對用戶數據變動的敏感程度:越敏感的數據越危險,越危險的數據越需要嚴謹的驗證機制。
\subsection{極致多因素認證}
傳統身份驗證系統中,多因素認證被廣泛使用,但僅被視為登入時的驗證方案。然而,我們認為多因素中的每個因素都可視為一個維度,而這些維度可由用戶自主選擇。因此,我們提出了極致多因素認證的概念:我們認為不存在特定的多因素驗證方法作為用戶登入的唯一方式,而應該反過來,從服務器的角度觀察,每個因素的驗證應該是用戶向服務器自主證明自己身份的方法。

因此,在足夠危險的情況下,可能需要連續多種因素的驗證;而在足夠安全的情況下,可能只需要一種因素的驗證。甚至即便遺失了某個重要因素,也不應被視為無法登入,而應被視為需要提供更多其他因素的驗證。
\subsection{半匿名制}
傳統身份驗證系統中,用戶身份通常不是實名制就是匿名制。然而,這種設計無形中造成了用戶隱私的困境。因此,我們提出了半匿名制的概念,讓用戶可以在關聯到真人的同時又可以拒絕被關聯到真人。半匿名機制旨在滿足 GDPR 中的被遺忘權與積極授權權,即用戶的個資只有在個人積極有意願提供時才能被使用。

解決遺忘權的設計理念是讓用戶自主掌握證明自己身份的權利。因此,只要用戶刪除了自己的身份,即使系統中仍有用戶的資料,也再無人可以證明資料的歸屬,因為唯一可以證明資料歸屬的用戶已刪除了自己的身份。

解決積極授權權的設計理念是讓用戶自主管理授權的證明。值得注意的是,這裡的用戶不只是服務的使用者,也可能是服務的提供者。因此,每個授權誕生的同時,都存在兩個證明:一個是用戶授權給服務的證明,另一個是服務接受授權的證明。這種設計可以解決現有系統中的部分問題。例如,當用戶發現個資被越權濫用時,可以出示簽章證明自己的權利,進而要求服務提供者負責。又或者,當服務提供者被指控越權濫用時,可以出示簽章證明自己的權利,進而避免被誤解。
\subsection{基於區塊鏈的共識機制}
在傳統身份管理系統中,基於安全與隱私的考量,數據孤島問題一直難以確實解決。在自主身份系統中,使用者可以用個人裝置存儲自己的身份資料,並且在加入另一個自主身份系統時,可以選擇性地分享部分資料。然而,這種設計無法徹底解決數據孤島問題。試想,如果用戶想共享的資料是有價值且可能被更改的,例如資產證明,這樣的資料不能僅靠用戶的裝置來存儲。

因此,我們提出了基於區塊鏈的共識機制。每個自主身份節點可以在鏈上提交針對特定用戶數據的校驗碼,進而讓其他節點可以驗證用戶特定數據的真實性。這種設計使用戶能在不同的自主身份系統中共享資料,同時保證資料取得共識。
\section{系統架構}
\subsection{整體說明}
\subsection{AID Server}
\subsection{Wallet}
\subsection{Consensus Core}
\section{資料結構}
\subsection{整體說明}
\subsection{AID Server}
\subsection{Wallet}
\subsection{Consensus Core}
\section{AID 的威脅模型}
\section{本章小結}
系統如何解決所有缺點,並且保留所有優點。