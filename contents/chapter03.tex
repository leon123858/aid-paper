% !TeX root = ../main.tex

\chapter{系統設計}
自主身份(AID)系統是一種創新的身份管理架構,其核心理念在於賦予個體對其身份資料的完全控制權。在這個系統中,三個關鍵角色相互協作:用戶、服務提供者和共識核心。用戶作為系統的終端使用者,通過個人設備全權管理自己的身份資料,根據需求選擇性地使用任意個服務並控制資料的流動。服務提供者則擔任用戶聚集的節點,不儲存用戶資料,而是根據用戶提供的資訊完成相應服務,並可靈活地根據實際場景客製化管理模式。共識核心則扮演著關鍵的中介角色,連接服務與服務、服務與用戶,提供用戶數據背書和信任評價機制,成為整個生態系統的基石。

本章節將深入探討此系統的架構設計、資料結構和潛在威脅模型,全面分析其如何在保障用戶自主權的同時,實現高效、安全的身份管理。
\section{系統的重要進步}
為實現用戶對其身份的完全自主權,本研究提出了多項創新機制,旨在解決現有的技術困境。
\subsection{別名優先機制}
傳統的身份驗證系統要求使用者提供唯一帳號作為登入識別。然而,這種設計無形中剝奪了用戶對個人名稱的自主權。一個連名稱都無法自主的系統,顯然難以達成本研究的設計目標。因此,我們提出讓用戶在註冊身份時優先使用自己喜歡的別名,再透過 UUID 機制\cite{uuid}產生唯一識別號。在日常操作中,系統優先讓用戶使用別名作為帳號,只有當別名因重複而難以識別時,才會要求用戶使用與 UUID 關聯的機制進行識別。當然,何時該使用別名,何時該使用 UUID,這是個複雜的問題。為此,我們提出了「基於用戶時空的分析方法」,試圖系統性地進行分析。
\subsection{基於用戶時空的分析方法}
每個身份本質上可視為一個隨時間變化的動態向量空間,其維度可謂無窮。每次對用戶的身份驗證都可視為取得特定時間僅含部分維度的向量。這方法的核心是讓用戶自主決定提供哪些維度,並在每次與系統互動時攜帶這些資訊。基於此概念,系統可藉由比對當次傳入的向量和近期暫存在快取中的所有向量來推測用戶的真實身份。

舉例來說:用戶在某次登入時提供了自己的別名、性別、所在地區等資訊,而在下次登入時僅提供了性別、所在地區等資訊,系統可以通過比對這兩次向量來推測用戶的別名。這種方法使用戶在不提供完整資訊的情況下,仍能通過部分資訊完成身份驗證。

然而,這種方法也帶來了一些問題。例如,用戶提供的維度多寡會影響系統的準確性,甚至用戶提供的維度是否包含可變資訊會影響系統的安全性。因此,我們提出維度的選擇甚至各個維度的權重應由用戶自主決定,而系統僅提供推薦機制。這樣可讓用戶在不同情境下使用不同方法,以滿足其需求。

總的來說,當僅有一個用戶被比較出來時視為可識別,有零個或多個用戶被比較出來時視為不可識別。這種機制受比較方法影響,因此我們提出了「基於危險程度的驗證機制」,希望能在不同情境下選擇適合的比較方法。
\subsection{基於危險程度的驗證機制}
此機制主要使用在比較身份與歷史紀錄時選擇適合的比較方法。不同的比較方法可能導致不同的結果,因其本質上代表了不同的嚴謹程度。在嚴謹程度較高的情況下,用戶可能需要更多維度符合,或需要在更接近的時間點內提供的紀錄才算數。這可能導致用戶難以找到識別對象,進而要求補充更多資訊再次驗證。相反,在嚴謹程度較低的情況下,用戶可能僅用少量維度的資訊配合較遠時間點的紀錄來完成驗證,這可能導致用戶被誤認為其他用戶,威脅系統安全。

考慮用戶行為的危險程度,我們設計出以下規則:
\begin{enumerate}
  \item 當用戶的行為被視為危險時,系統應提高驗證的嚴謹程度。
  \item 當用戶的行為被視為安全時,系統應降低驗證的嚴謹程度。
\end{enumerate}

接著,我們將嚴謹程度拆分為時間和空間兩個軸,使用笛卡爾座標系統表示,以細分出更複雜的情境:
\begin{enumerate}
  \item 當用戶行為超危險時,比較標準應為時間非常近或多個維度符合。
  \item 當用戶行為危險時,比較標準應為時間相對近或維度相對符合。
  \item 當用戶行為安全時,比較標準應為時間相對遠或維度相對不符合。
  \item 當用戶行為超安全時,比較標準應為時間非常遠或僅幾個維度符合。
\end{enumerate}

最後,用戶行為的危險程度在我們的系統中是由用戶自主決定的,系統僅提供推薦機制。我們希望藉此滿足所有用戶在各種情境下的需求。例如,對於個人銀行帳戶,可能所有行為都視為最高危險,因此需要最高的驗證嚴謹程度。而對於個人社群帳戶,讀取文章的行為可能視為不危險,發布文章則視為危險,因此需要不同的驗證嚴謹程度。我們建議用戶將危險的概念定義為對用戶數據變動的敏感程度:越敏感的數據越危險,越危險的數據越需要嚴謹的驗證機制。
\subsection{極致多因素認證}
傳統身份驗證系統中,多因素認證被廣泛使用,但僅被視為登入時的驗證方案。然而,我們認為多因素中的每個因素都可視為一個維度,而這些維度可由用戶自主選擇。因此,我們提出了極致多因素認證的概念:我們認為不存在特定的多因素驗證方法作為用戶登入的唯一方式,而應該反過來,從服務器的角度觀察,每個因素的驗證應該是用戶向服務器自主證明自己身份的方法。

因此,在足夠危險的情況下,可能需要連續多種因素的驗證;而在足夠安全的情況下,可能只需要一種因素的驗證。甚至即便遺失了某個重要因素,也不應被視為無法登入,而應被視為需要提供更多其他因素的驗證。
\subsection{自主簽章}
本研究提出了一種創新的基於區塊鏈技術的自主身份驗證流程,旨在解決傳統身份管理系統中使用者對身份驗證缺乏自主權的問題。在傳統模式下,使用者的身份驗證資訊通常由身份服務提供者集中管理,這種做法實質上限制了使用者對其個人身份資訊的控制權。為了解決這一問題,我們參考了Tze-Nan\cite{NTU202102846}提出的自主簽章機制,設計了一個基於區塊鏈技術的新型身份管理流程。

本研究提出的身份管理流程包含以下關鍵步驟:
\begin{enumerate}
  \item 隱私簽章:使用者可以針對相同的自主身份在不同的簽章者處建立不同目的的簽章,其內提供不同數據與權限。
  \item 區塊鏈驗證:簽章的簽名會被簽章者記錄在區塊鏈上,服務提供者可以在區塊鏈上查詢簽章的真實性,確保身份資訊的可信度。
  \item 簽章撤銷:使用者可以隨時撤銷簽章,並在區塊鏈上提交撤銷的簽名。這種機制可以應用於簽章遺失等情況,提高系統的安全性和可靠性。
  \item 自主註冊:使用者在註冊時,主動向服務提供者提交自己的簽章,表明自己的身份。
  \item 彈性驗證:使用者在進行身份驗證時,可以根據簽章自主選擇多因素驗證(MFA)的方式,提高身份驗證的便利性和安全性。
\end{enumerate}
此外,本系統的簽章機制支援多樣化的功能,進一步提升了使用者對其身份資訊的控制權:
\begin{itemize}
  \item 自定義多因素認證選項
  \item 選擇性資訊揭露
  \item 設置簽章有效期限
  \item 指定特定的驗證條件 (如特定設備,地點或時間)
  \item 指定特定的驗證規則(設備和網路使用限制)
  \item 資源存取權限控制 (如檔案存取權限)
\end{itemize}

總結來說,通過這種機制提升了系統對用戶隱私和安全的保護程度,讓用戶能夠更自主地管理其身份驗證流程,保持對個人身份驗證的控制權。
\subsection{數據共識}
在傳統身分管理系統中,基於安全和隱私的考慮,資料孤島問題一直難以有效解決。自主身分系統的出現為此提供了解決方案。使用者可以使用個人設備儲存自己的身份訊息,並在加入另一個自主身份系統時選擇性地上傳部分資訊。但這種設計仍然無法徹底解決資料孤島問題。尤其是當使用者想要共享的資料價值較高且可能被更改時,例如資產證明,此類資料無法僅在使用者的個人裝置上儲存和操作。

為了解決這一問題,本研究提出了一種基於區塊鏈的共識機制。在這種機制中,每個自主身份服務可以在區塊鏈上提交針對特定使用者特定數據的校驗簽章,進而讓其他服務能夠驗證使用者特定數據的真實性。這種設計使用戶能夠在不同的自主身份服務中安全地共享資料,同時向其他用戶保證資料的一致性和可信度。

這種基於區塊鏈的共識機制不僅有效解決了數據孤島問題,還提供了多方面的優勢。首先,通過區塊鏈技術確保了數據的完整性,有效防止數據被篡改。其次,它實現了使用者在不同自主身份系統間的無縫數據共享。再者,這種機制仍然保護了使用者的隱私,讓使用者能夠控制哪些數據被共享,保持對個人資訊的自主權。最後,它還支援實時驗證,其他服務可以即時驗證使用者數據的真實性,無需複雜的跨系統認證流程。
\subsection{自主身份數據管理}
基於前文所述的「自主簽章機制」與「數據共識機制」,我們發展出了「自主身份數據管理」方法,旨在徹底解決被遺忘權和積極數據授權等棘手的隱私問題。在自主身份框架下,用戶通過自有裝置保存個人數據,包括簽章與資料。當用戶需要證明身份時,需依照「自主簽章機制」上傳簽章給服務提供者,以完成對用戶的信任與驗證。而當用戶需要使用數據時,則上傳本地的相關數據,並可透過「數據共識機制」使服務提供者信任用戶所提供數據的真實性。

此外,針對特殊的隱私問題,自主身份數據管理提供了以下解決方案:
\begin{itemize}
  \item \textbf{被遺忘權}:當用戶希望遺忘數據時,可直接清除個人保留的數據。區塊鏈中僅保留簽名,而不保留數據本身。理論上,服務內部不會存儲用戶數據;即使確實存儲了,由於唯一能證明數據擁有人的是用戶本身,因此相當於用戶與數據無關。這個概念類似於 Cameron\cite{cameron2005laws}所描述的單向身份:用戶可以通過個人證明指向自己的數據,但僅有數據無法指向用戶。
  \item \textbf{積極數據授權}:採用類似「自主簽章」的方法,使用戶與服務提供者對數據授權產生明確共識。當用戶授權服務提供者使用數據時,會對數據與使用範圍生成簽章,並上傳對應簽名至區塊鏈,然後將簽章與數據傳送給服務提供者。之後,用戶可利用公開簽章證明數據被濫用,反之,服務提供者也可利用公開簽章證明數據被合法使用。這樣的設計使用戶與服務提供者之間的數據授權變得更加明確且公平。
\end{itemize}

總的來說,自主身份數據管理方法提供了一個數據管理方法,讓用戶能夠更好地控制自己的數據,保護自己的隱私,並讓他人信任數據的真實性。\newpage
\section{系統架構}
自主身份系統如圖\ref{fig:aid-layers}所示,從宏觀來看可以被分成三個層次:由上而下分別是共識層、服務層與數據層。共識層負責確保數據的共識,服務層負責提供各種服務,數據層負責存儲用戶數據。
\begin{figure}
  \centering
  \includegraphics[width=0.8\textwidth]{figures/aidLayers.png}
  \caption{自主身份系統分層架構}
  \label{fig:aid-layers}
\end{figure}

更深入地看,共識層的核心組成是具備共識機制的區塊鏈系統。這一層級可通過跨鏈橋等先進機制實現水平擴展,顯著提升系統的可擴展性。區塊鏈中的智能合約扮演著關鍵角色,它為服務層和數據層提供了對共識數據進行讀寫操作的介面。值得注意的是,共識層的實現並不局限於區塊鏈技術。在存在可信第三方機構的情況下,其他形式的共識機制同樣可以被採用,這為系統設計提供了更大的靈活性。

服務層由多樣化且相互獨立的網路服務構成,這些服務可能以移動應用、網站或API服務等形式呈現。儘管各服務的具體需求可能大相徑庭,但它們都需要強大的身份管理功能。為了滿足這一普遍需求,我們提出了一個統一的解決方案:AID Server。我們設計了一個用於多種程式語言的後端SDK規格,它在保留客製化空間的條件下為服務提供者提供了一個標準化的身份管理介面。通過AID Server,服務開發者能夠輕鬆地實現與共識層和數據層的無縫對接,大大簡化了開發流程並維持了系統的一致性。

數據層主要由大量獨立的終端應用組成,這些應用可能是智能手機APP、個人電腦軟體或物聯網(IoT)設備等。與服務層類似,數據層中的應用雖然功能各異,但都需要可靠的身份管理能力。針對這一需求,我們設計了名為Wallet的前端SDK規格。Wallet能在多種程式語言下為應用開發者提供統一的身份管理介面。這使得開發者能夠輕鬆地實現數據層應用與共識層和服務層的有效對接,從而構建出一個完整且高效的生態系統。

這種多層架構設計不僅確保了系統各部分的模組化和解耦,還提高了整體系統的可擴展性、靈活性和安全性。通過標準化的介面和SDK規格,我們大大降低了開發難度,同時提高了不同層級間的互操作性。接著,我們將進一步介紹各層的的具體規格與設計細節。
\subsection{Consensus Core}

\subsubsection{模塊規格}
\subsubsection{模塊設計細節}
要聊到區塊鏈治理的問題
\subsection{AID Server}
\subsection{Wallet}
要聊到數據存儲的問題

\section{資料結構}
\subsection{AID Server}
\subsection{Wallet}
\subsection{Consensus Core}
\section{自主身份系統的威脅模型}
\section{本章小結}
系統如何解決所有缺點,並且保留所有優點。