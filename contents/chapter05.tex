% !TeX root = ../main.tex

\chapter{結論與未來展望}
本研究深入探討了自主身份(AID)系統,這是一個旨在解決現有身份管理系統挑戰的創新解決方案。通過將「自主」的概念融入數位身份管理,AID系統致力於在保障道德標準的同時,賦予用戶對其身份信息和數據的完全控制權。
\section{研究結論}
儘管AID系統通過結合無數創新的機制與系統架構,已展現出巨大的潛力,但作為一個新興的研究領域,仍有許多方面值得未來深入探討:
\begin{enumerate}
    \item \textbf{艱鉅的推廣任務:} AID系統可以說是一個全新的身份管理系統,需要克服用戶習慣、技術標準、法律法規等多方面的障礙。未來需要進一步研究如何推廣 AID 系統,提高用戶接受度和市場競爭力。
    \item \textbf{更高的開發難度:} AID系統不但使用了區塊鏈等新技術,還引入了反轉數據層等新概念,這將對開發人員的技術水平提出更高的要求。未來需要進一步研究如何降低開發難度。
    \item \textbf{大規模應用與部署:} 目前,AID系統的概念驗證主要集中在AI聊天服務和支付系統等有限的場景。未來需要進一步研究如何將AID系統應用於更廣泛的領域,例如金融、醫療、政務等,並探討大規模部署的可行性和挑戰。
\end{enumerate}
\section{未來展望}
我們將從以下幾個領域展望AID系統的未來發展:
\begin{itemize}
    \item 金融
    \item 醫療
    \item 政務
\end{itemize}
這些領域都是當前身份管理系統存在問題的重要應用場景,AID系統的應用將為這些領域帶來深遠的影響。
\section{總結}
總而言之,自主身份系統代表了數字身份管理的未來方向,它為構建一個更加安全、自由、便捷和公平的數位社會提供了巨大的潛力。相信通過持續的研究和創新,AID 系統將在未來展現出更廣闊的應用前景,並為人類社會帶來深遠的影響。