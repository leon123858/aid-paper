% !TeX root = ../main.tex

\chapter{結論與未來展望}
本研究深入探討了自主身分(AID)系統,這是一個旨在解決現有身分管理系統挑戰的創新解決方案。通過將「自主」的概念融入數位身分管理,AID系統致力於在保障道德標準的同時,賦予使用者對其身分信息和數據的完全控制權。研究過程中,本研究不僅提出了理論框架,還進行了系統設計和概念驗證,為未來的實際應用奠定了基礎。

本研究的主要貢獻包括:提出了自主身分的完整概念、設計了自主的身分管理機制、實現了高度的安全性和隱私保護、優化了使用者體驗、確保了法規遵循,並通過概念驗證展示了系統的可行性。

儘管AID系統展現出巨大潛力,但作為一個新興的研究領域,仍面臨諸多挑戰:
\begin{enumerate}
    \item \textbf{艱鉅的推廣任務:} AID系統可以說是一個顛覆性的身分管理系統,需要克服使用者習慣、技術標準、法律法規等多方面的障礙。未來需要進一步研究如何推廣 AID 系統,提高使用者接受度和市場競爭力。
    \item \textbf{更高的開發難度:} AID系統不但使用了區塊鏈等新技術,還引入了反轉數據層等新概念,這將對開發人員的技術水平提出更高的要求。未來需要進一步研究如何降低開發難度。
    \item \textbf{大規模應用與部署:} 目前,AID系統的概念驗證主要集中在AI聊天服務和支付系統等有限的場景。未來需要進一步研究如何將AID系統應用於更廣泛的領域,例如金融、醫療、政務等,並探討大規模部署的可行性和挑戰。
\end{enumerate}

展望未來,AID系統有望在多個領域帶來革命性變革。在金融領域,它可以實現跨機構的無縫身分驗證,簡化開戶和貸款流程,並建立更透明、公平的個人信用評估系統。在醫療領域,AID系統可以幫助建立患者完全掌控的電子病歷系統,實現跨機構就醫的同時保護患者隱私。在政務服務方面,基於AID的電子身分證系統可以實現一站式政務服務,而安全透明的電子投票系統則有助於提高公民參與度。

自主身分系統代表了數位身分管理的未來,有望成為下一個數位時代的代表性技術。儘管面臨挑戰,但通過持續研究和外部合作,AID系統將展現更廣闊的應用前景。未來研究將聚焦於大規模實際應用、使用者推廣、與現有系統的融合,以及系統性能和安全性的優化。本研究期望更多人參與AID系統的探索和實踐,共同推動這一技術的發展,為數位社會的進步做出貢獻。