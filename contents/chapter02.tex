% !TeX root = ../main.tex

\chapter{文獻探討}
身份系統的發展經歷了從中心化到去中心化的演變,現代身份系統設計面臨著用戶體驗、認知負擔、隱私保護和平等信任等多方面的挑戰,需要同時滿足技術創新、法規遵循和行業最佳實踐,以創造出既能滿足多樣化需求又具備足夠靈活性的解決方案,本章將探討相關文獻以深入理解這些挑戰及可能的解決途徑。
\section{身份系統的想像}
Internet發源於20世紀70年代,逐漸發展成為現代網際網路的基礎架構。最初,Internet的設計源自美國國防部的ARPANET計畫,旨在滿足軍事需求下的網路可用性和穩定性。因此,其初期設計基於以下假設\cite{Pekka2010HIP}:
\begin{itemize}
  \item 最終使用者至少在最低程度上相互信任。
  \item 網路由於潛在的物理攻擊而本質上不可靠。
\end{itemize}
這些假設在當時的網路環境中是合理的。然而,隨著網路的普及和應用範圍的擴大,這些假設已不再適用\cite{tomorrowinternet}。Internet從一個研究驅動的項目演變為主流社會的重要組成部分,新的需求不斷湧現,不僅挑戰了原有的設計原則,還促使我們重新審視一些既有的原則。

在現代網路環境中,使用者之間的信任關係變得日益複雜。人們需要可靠的方式來識別自己和他人,以便在網路上進行交流和交易。因此,各種身份系統應運而生,以滿足網路環境中的身份識別需求。儘管人們希望找到一種統一而理想的身份管理系統,但正如Cameron\cite{cameron2005laws}所指出的,這樣的系統實際上並不存在。身份系統涉及的範疇廣泛而複雜,個人與組織之間存在多樣化且往往相互衝突的需求,試圖通過單一標準來限制或規範這些需求是不切實際的。

例如,終端用戶可能希望自由地訪問和分享信息,而內容提供商和知識產權持有者則希望保護其知識產權。政府可能希望監管某些網絡活動以維護社會秩序,而用戶和隱私倡導者則強調個人隱私的重要性。這種複雜的利益衝突導致了諸如網絡中立性、數據隱私、內容審查等一系列熱點問題的出現\cite{Wu2003NetworkNeutrality}。

面對這種複雜的網絡環境,Blumenthal\cite{Blumenthal2001RethinkingThe}認為確保設計上的一般性、彈性與開放性至關重要。他設想的未來充滿衝突:企業的管理者和被管理者爭論分紅,垃圾郵件的發送者和接收者爭論各自的困難。在這個網路世界中,不同身份的參與者沒有絕對的贏家,也沒有天生的失敗者。理想的系統應該通過所有用戶不斷的爭論和互動逐漸形成。

這種新的設計思路不僅需要考慮技術因素,還需要權衡經濟、社會、法律等多方面的因素。正如Lessig\cite{lessig2000}所言,"程式碼就是法則"("Code is Law")。軟體設計本身就在表達某種價值觀,人們需要縝密的思考,結合對未來世界的想像,才能創造出足以改變網路環境既有困境的身份系統。

總的來說,現代網路環境的變遷對身份系統提出了新的挑戰,這些挑戰不僅來自技術層面,還涉及社會、文化和法律等多個方面。為了應對這些挑戰,我們需要重新思考身份系統的設計理念,尋找一種既能適應當前多樣化需求,又能為未來發展預留足夠靈活性的解決方案。
\section{身份系統的迭代}
身份系統的設計經歷了多個階段的演變,每個階段都試圖解決特定的問題,同時也帶來了新的挑戰。本節將介紹不同世代的身份系統設計,以說明彼此衝突的需求和技術限制,並為後續討論提供背景。
\subsection{中心化身份}
中心化身份系統是最早期的身份管理方案,在企業和政府機構中廣泛應用。典型例子包括 Windows Active Directory 和 LDAP(輕量級目錄訪問協議)\cite{microsoft2021active, sermersheim2006lightweight}。這類系統的主要優勢在於其集中管理的特性,便於系統管理員進行用戶管理和權限控制,同時確保組織內部身份信息的一致性和及時更新。然而,中心化身份系統也面臨著諸多挑戰,如單點故障風險、隱私保護問題,以及跨組織可遷移性差等。用戶通常需要為每個服務創建單獨帳戶,這不僅增加了認知負擔\cite{josang2007security},還導致用戶身份被服務提供商完全控制,缺乏自主權。
\subsection{聯合身份}
主要為了解決同一個用戶擁有太多身份的認知負擔,聯合身份系統允許不同組織間共享身份信息,代表性技術包括 SAML(安全斷言標記語言)和 WS-Federation \cite{oasis2005security, goodner2009web}。這種模式的出現大大改善了用戶體驗,實現了單點登錄(SSO),減少了密碼疲勞問題。聯合身份系統促進了組織間的協作和資源共享,同時也降低了重複身份管理的運營成本。然而,這種模式也帶來了隱私方面的挑戰\cite{ahn2007user},如用戶信息在多個服務提供商間共享可能違反 GDPR 等隱私法規 \cite{GDPR2016}。此外,實施和維護聯合身份系統的技術複雜度較高,參與組織之間需要建立並維護信任關係。
\subsection{使用者中心的身份}
為了解決隱私方面的問題,使用者中心的身份系統逐漸興起。OpenID 和 OAuth 等協議的出現\cite{sakimura2014openid, hardt2012oauth},標誌著身份管理向用戶賦權的重要轉變。這種模式增強了用戶對個人身份信息的控制權,提供了更大的靈活性,允許用戶選擇不同的身份提供者。使用者中心的身份系統實現了以服務供應商為單位的資訊範圍控制,在一定程度上改善了隱私保護。然而,這種方式也面臨著身份碎片化的問題,多個身份提供者的存在可能導致用戶體驗的不一致。安全風險如釣魚攻擊和身份提供者數據洩露等仍然存在\cite{sun2012devil}。此外,儘管用戶獲得了更多控制權,但他們仍然在某種程度上依賴中心化的身份提供者,因此用戶的自主性還擁有很大的提升空間\cite{allen2016selfsovereign}。
\subsection{去中心化身份}
去中心化身份是身份系統概念的最新發展\cite{preukschat2021self},其代表性例子包括基於區塊鏈的 uPort 和微軟的 ION(去中心化身份覆蓋網絡)\cite{lundkvist2017uport, microsoft2020ion}。這種方式賦予用戶對自身身份和個人數據的完全控制權,同時提高了身份的可攜性和一致性。通過利用區塊鏈技術,去中心化身份系統不僅增強了用戶與服務提供商之間的平等性,還提供了抗審查的特性。
然而,作為一種新興技術,去中心化身份系統也面臨著諸多挑戰\cite{s22155641}。首先,它與現有法律框架可能存在潛在衝突,例如與 GDPR 中的被遺忘權不相容\cite{finck2018blockchains}。其次,去中心化身份系統的技術複雜性可能影響普通用戶的使用體驗,降低其易用性\cite{kubach2020self}。此外,還有諸如隱私保護、系統互操作性等問題需要解決。
\subsection{未來展望}
綜上所述,身份系統的發展經歷了多個階段,每個階段都試圖解決特定的問題。從中心化到去中心化,身份系統的設計逐漸向用戶賦權,提高了用戶對自身身份的控制權。然而,即使到了去中新化階段,我們依舊不認為找到了理想的解決方案。如同Schardong等人\cite{s22155641,soltani2021surveydid}所說,當今的去中心化身份系統仍面臨著許多挑戰,包括安全性、隱私保護、易用性、信任建立等問題。因此,我們認為身份系統的設計仍有很大的改進空間,需要更多的研究和實踐來不斷優化。
\section{自主身份的設計原則}
為了設計出一個理想的身份系統,在本節中,我們會從不同維度的多個方向來探討身份系統的設計原則。這些原則旨在幫助我們理解身份系統的核心特徵,並為未來的設計提供參考。
\subsection{用戶體驗}
用戶體驗在身份系統設計中扮演著關鍵角色,直接影響系統的可用性和採納率。然而,Hamme等人\cite{inproceedings}的研究闡明了用戶體驗、安全性和隱私保護之間的複雜關係。該研究指出了一個普遍存在的現象:用戶傾向於選擇最簡單的方式來設置和使用身份系統,這種傾向可能導致系統安全性和隱私保護程度的降低。

這種情況產生了一個兩難困境,為了提高安全性而強制用戶採用複雜的身份驗證方式可能會適得其反。例如Zhang等人\cite{zhang2010security}的研究表明,要求用戶定期更改密碼往往導致用戶僅修改特定字元,反而造成更大的安全隱患。同樣地,為了增強隱私保護而要求用戶完成詳細的隱私設置也可能降低用戶體驗。Acquisti等人\cite{acquisti2017nudges}的研究發現,複雜的隱私設置過程往往使用戶感到困惑和沮喪,甚至導致他們放棄設置而選擇默認選項,從而降低了隱私保護水平。

為解決這一困境,研究者提出了「無摩擦驗證」(Zero Friction Authentication)的概念,旨在最小化用戶在設置和使用過程中遇到的困難,同時維持適當的安全性和隱私保護水平。Hamme等人\cite{inproceedings}強調,無摩擦驗證的核心目標是在保護用戶安全和隱私的同時,顯著降低用戶的操作負擔。這種平衡對於現代身份系統的設計至關重要,因為它直接影響系統的使用率和效能。

為實現無摩擦驗證,近年來安全領域出現了多種新技術。如Ghorbani等人\cite{ghorbani2020fido2}研究了無密碼登入(如FIDO2)的可用性,發現這種方法能透過硬體密鑰在手機上跨裝置完成高安全性的驗證,且使用者普遍認為方便並願意持續使用。Wiefling等人\cite{wiefling2021rba}探討了基於風險的驗證(RBA),該方法透過追蹤身份與系統互動的歷史數據,在每次服務請求時動態判斷危險性,並在危險時採用更安全的多因素驗證(Multi-Factor Authentication, MFA)\cite{bonneau2012mfa}。Alaca等人\cite{alaca2016devicefingerprinting}關於裝置指紋的研究顯示,透過在每次用戶對服務發出請求時記錄並比對裝置指紋,可以有效辨識部分惡意行為,且不會增加用戶的操作負擔。

另外,為解決複雜隱私設置帶來的問題,Acquisti等人\cite{acquisti2017nudges}提出了「隱私設計」(Privacy by Design)的概念。這種方法將複雜的設置過程分解為多個簡單步驟,並在用戶使用系統的不同階段逐步引導用戶完成設置。研究表明,這種方法不僅能提高用戶的隱私保護水平,還能顯著改善用戶體驗。

這些新技術的應用表明,在不影響用戶體驗的前提下提供更高的安全性和更好的隱私保護是可能的。然而,如何在自主身份系統中實現真正的無摩擦驗證,以及如何有效地平衡用戶體驗、安全性和隱私保護的需求,仍然是一個值得深入研究的課題。
\subsection{用戶認知}
用戶認知在數位身份管理中扮演著至關重要的角色,直接影響到系統的安全性和有效性。LastPass\cite{lastpass2020psychology}的研究揭示了用戶認知與實際情況之間存在顯著差距:用戶平均估計自己擁有20個線上帳號,而實際上平均擁有37個以上的帳號。這種認知偏差背後反映了用戶使用上的不便,例如經常因為忘記密碼而無法登入帳號,或者因為各個帳號資料不互通而需要耗費大量時間來管理。

Dhamija等人\cite{dhamija2008sevenflaws}的研究進一步指出,用戶對身份管理系統的認知和理解程度直接影響其安全行為。隨著需要管理的用戶名和密碼數量增加,用戶往往感到困惑,進而採取不安全的行為,如使用弱密碼或在多個平台使用相同密碼等。此外,用戶對身份管理系統的認知不足也會導致他們無法有效應對釣魚攻擊、社交工程等安全威脅。

為了解決這個問題,未來的身份管理解決方案應該朝多個方向發展。首要任務是簡化多層次、多維度的用戶身份管理,允許單一的身份管理多樣的別名,以適用於不同的場景。例如,用戶可以用唯一的帳戶創建三個別名,分別對應自己的三種社會身份:在家中是家長,在工作中是員工,在社交場合是朋友。甚至針對單一的服務,用戶也可以擁有多種別名,如在論壇中既可以以專家身份發表權威言論,也可以作為普通用戶表達個人觀點。這樣的設計可以幫助用戶在盡可能不增加認知負擔的情況下有效管理自己的身份。

然而,真正簡化用戶認知並非易事。即使是宣稱已解決這個問題的使用者中心身份系統,實際上也未能完全做到。以 Google 的組織管理文件\cite{gcp2024identity}為例,為了確保不同組織擁有不同的安全限制與規定,系統仍然要求用戶在不同組織間創建不同的身份。這表明,同時簡化用戶認知與滿足組織需求,仍然是一個有待解決的挑戰。
\subsection{隱私保護}
在數位時代,隱私保護已成為身份系統設計的核心考量之一。歐盟制定的《通用數據保護條例》(GDPR)\cite{GDPR2016}代表了目前全球最嚴格的隱私保護標準。本研究認為,一個理想的身份系統應當能夠全面符合 GDPR 的要求,從而確保用戶隱私得到最大程度的保護。然而,近年來的 GDPR 違規案例表明,即便是大型企業也面臨著遵守某些 GDPR 規定的挑戰。

基於Schardong等人的研究\cite{s22155641},我們發現當前身份系統中存在兩個尤為突出的關鍵問題。首先是用戶積極授權的實現困難。Saemann\cite{saemann2022investigating}的研究強調,在當前的身份系統框架下,企業難以實現用戶對數據使用的明確授權。具體而言,企業難以證明其對數據或權限的使用行為已獲得用戶授權,而用戶也缺乏有效途徑證明自己的數據或權限被不當使用。這種情況不僅增加了企業的法律風險,也削弱了用戶對系統的信任。

第二個問題是被遺忘權的實現困難。Smirnova\cite{smirnova2024understanding}指出,滿足用戶的被遺忘權在當前身份系統中存在著一定的挑戰。用戶數據在系統中往往呈分散狀態,即使刪除核心用戶的資料,仍可能保留用戶的系統日誌或與其他用戶的互動數據。這種情況使得完全實現被遺忘權變得複雜而困難,可能導致用戶隱私無法得到全面保護。

基於以上分析,我們提出符合現代隱私保護要求的身份系統應該具備兩個關鍵特點。首先,系統應提供合理的機制,使用戶或企業能夠證明其數據使用行為是否符合授權,這將有助於提高系統的透明度和可信度。其次,系統應提供有效的方法讓用戶行使被遺忘權,確保用戶不會被難以刪除的數據綁架。這意味著系統需要設計更精細的數據管理和刪除機制。在後續研究中,我們將詳細探討如何在自主身份系統中實現這些特性,並提出相應的技術解決方案。
\subsection{平等信任}
身份系統中的平等信任問題是一個複雜的多方利益平衡問題,涉及系統的公平性和可信度。研究\cite{preukschat2021self}強調了身份系統中各方利益的衝突,主要表現在用戶之間的權益差異、不同系統間的互操作性問題,以及用戶與系統供應者之間的利益衝突。例如,身份系統供應者可能希望獲取更多用戶個人資料以獲取利益,而用戶則希望保護自己的隱私。這種利益衝突如果處理不當,可能導致系統環境惡化、用戶權益受到侵犯,以及市場壟斷和不公平競爭。Zuboff\cite{zhang2010security}的研究進一步指出,這種數據收集和利用的不平等可能導致所謂的「監視資本主義」,對個人自由和社會公平造成深遠影響。

在當前的身份系統中,建立健全的信任模型仍然是一個重大挑戰。傳統的二元邏輯驗證模式(即完全信任或完全不信任)已不能滿足現代身份系統的需求。研究\cite{s22155641}指出,現實世界的信任往往是模糊而不確定的,人們很難用簡單的真假邏輯分辨用戶驗證的成功與否。例如,可能存在多組憑證同時存在,部分驗證成功,部分驗證失敗的情況,或者不同憑證的可信度和重要性各不相同。Josang等人\cite{josang2006exploring}提出的主觀邏輯數學框架為處理這類多組不確定性憑證的問題提供了一個系統性的解決方案。該框架將憑證的可信度表示為一個區間,從而更好地模擬了現實世界的身份驗證過程。

身份驗證的雙向性問題,尤其在去中心化系統中表現突出,正如研究\cite{dhamija2008sevenflaws}所強調的。用戶需要向系統證明自己的身份,同時系統也需要向用戶證明自己的合法性和可信度。其中,雙向的合法性可以簡單透過如SSL之類的驗證協議完成。但雙向的信任建立過程則可能需要建立一個多維度、連續性的信任評估機制,以滿足不同場景下的信任需求。Chohan等人\cite{chohan2024decentralized}的研究提供了DAO(Decentralized Autonomous Organization)治理的核心概念,可以為實現這些特性提供參考。

在後續的研究中,我們將探討如何在自主身份系統中實現技術上的互相驗證與制度上的互相信任,最終構建一個能夠平衡各方利益、促進公平競爭的自主身份生態系統。這種系統將能夠在保護用戶權益的同時,也為系統供應者提供合理的發展空間,從而實現真正的平等信任。
\section{自主身份系統的評估}
鑒於當前身份系統在各個方面面臨的特定挑戰,我們提出以下評估標準,用以衡量自主身份系統的設計是否符合下一代身份系統的要求:
\begin{itemize}
  \item 創新突破:自主身份系統的設計應符合上節所述原則,才能實現真正的技術創新。以下陳列:
        \begin{itemize}
          \item 用戶體驗:提供無摩擦驗證,簡化操作流程,提高系統易用性。
          \item 用戶認知:提供簡單的身份管理機制,幫助用戶有效管理身份,減少認知負擔。
          \item 隱私保護:實施有效的隱私保護機制,保護用戶個人數據,確保隱私得到充分保障。
          \item 平等信任:建立互相驗證和互相信任的機制,促進所有個體間平等的信任關係。
        \end{itemize}
  \item 法規遵循:自主身份系統應符合相關法律法規,特別是在數據保護和隱私保護方面。例如:
        \begin{itemize}
          \item GDPR:遵守一般資料保護規範\cite{GDPR2016}的要求,包括用戶同意、數據可控性和被遺忘權等規定。
          \item NIST:符合美國國家標準\cite{NIST800-63-3},涵蓋多因素驗證、風險評估、身份驗證和授權等要求。
        \end{itemize}
  \item 公認原則:遵循身份識別領域的公認原則,包括但不限於:
        \begin{itemize}
          \item 身份法則:符合Kim Cameron提出的身份管理基本法則\cite{cameron2005laws},包括用戶控制、最小化披露、合法性和互操作性等原則。
          \item 避免常見缺陷:克服Dhamija等人\cite{dhamija2008sevenflaws}指出的身份管理七大缺陷,如密碼管理、安全提示和密碼重置等問題。
        \end{itemize}
\end{itemize}
這些評估標準綜合考慮了技術創新、法律合規性和業界最佳實踐,為評估和設計下一代自主身份系統提供了全面的合理的目標。
\section{總結}
本章從身份系統的迭代、設計原則和技術評估三個方面探討了現代身份系統的設計問題。我們發現,現代身份系統的設計仍面臨著諸多挑戰,包括用戶體驗、用戶認知、隱私保護、平等信任等方面。為了解決這些問題,我們提出了一系列設計原則,包括無摩擦驗證、隱私設計、用戶認知簡化、......等。在未來的研究中,我們將進一步探討如何在自主身份系統中達到這些評估標準,並提出相應的技術解決方案。