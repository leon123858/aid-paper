% !TeX root = ../main.tex

\chapter{文獻探討}
\section{身份系統的困境}
隨著網路技術的發展,數位化身份的議題日益受到重視。然而,核心的問題在當前的網路系統在設計之初並未考慮到身份管理的問題\cite{cameron2005laws}。進而多數身份系統基於軟體強硬的拼接出各式各樣的臨時身份層作為身份驗證的基礎,這也成為了身份技術難以跨越的現實問題。
其實假設性的正解是存在的,就如同網路服務中的 TCP/IP 協議一樣,我們期待著一個能夠解決身份管理問題的統一標準,人們可以因此基於軟體建構出統一的身份系統。然而,這樣的期待顯然不現實,身份系統的世界不像網路傳輸,人與人之間,組織與組織之間有截然不同的需求,要求人們限制自己對身份的選擇或需求是不合理的。
綜上所述,我們認為當前的身份系統存在著許多問題,這些問題不僅是技術上的,也是社會上的。我們認為引入自主的概念是解決現有問題的一個重要方向。因為我們可以在分開來滿足所有人與組織需求的同時讓他們漸漸地達成共識,進而以鬆耦合的型態構建出一個全新而統一的身份系統。
\section{身份技術的評量框架}
為了分析 AID 系統是否確實解決了現有身份系統的問題,我們需要找到一個評量框架,這樣我們才能夠對兩者進行客觀的比較。遺憾的是,目前尚未有一個公認完整的評量框架可以應用於身份系統的比較。因此,我們在依循幾個重要標準(\cite{GDPR2016},\cite{DIDCore},\cite{NIST800-63-3})的基礎上,提出了一個身份系統的評量框架。
關這個框架關於六個面向,分別是用戶體驗、用戶認知、隱私保護、安全驗證、平等自由與健全信任。這些面向是我們認為身份系統應該要具備的基本特徵,也是我們將用來評估 AID 系統的標準。
\subsection{用戶體驗}
論文\cite{inproceedings}清楚的說明了用戶體驗, 安全, 隱私的現狀, 用戶總會選擇最簡單的方式設置身份系統,這將導致過低的安全性與隱私性。但為了確保用戶安全而強迫用戶使用複雜的身份驗證方式,或是為了確保用戶隱私而要求用戶完成詳細設置,又會導致用戶體驗的下降。為此需要有新的驗證概念被稱做無摩擦驗證(Zero Friction Authentication), 能在不會太影響安全與隱私的條件下盡可能的降低用戶設置與使用的困擾。
\subsection{用戶認知}
密碼管理器供應商在 2020 年的報告\cite{lastpass2020psychology}揭露了用戶認知問題的嚴重性。平均來說,用戶認為自己擁有 20 個帳號,但實際上他們擁有 37 個以上的帳號,這些超出認知範圍的身份將導致用戶無法有效的管理這些帳號,不僅是便捷性很低,而且安全性也會受到威脅。我們認為用戶即便有了很多工具來幫助他們管理密碼,但用戶的認知中就有其限制。所以必須要有一個更為簡單的方式來管理多種而多層次的用戶身份。
\subsection{隱私保護}
GDPR 是一個由歐盟制定的有史以來最嚴格隱私法規。我們認為一個好的身份系統應該要能夠完全符合 GDPR 的要求,這樣才能夠保證用戶的隱私得到充分的保護。然而,參考近年來的 GDPR 違規案例,我們發現即便是大型的企業也無法完全遵守 GDPR 的部分要求。
其中論文\cite{saemann2022investigating}強調了用戶或企業難以保障用戶的積極授權,亦即用戶要明確授權系統使用自己的數據,數據才會被使用。因為當前的身份系統下,企業難以提出適合的方法明確的向用戶要求最小且必須的權限,用戶也難以證明自己的數據或權限被不當使用。另外,論文\cite{smirnova2024understanding}提到了被遺忘權, 即用戶可以要求系統供應商明確完整移除用戶的所有數據,難以被滿足的原因在於系統內關於用戶的數據往往是分散的,即使刪除了用戶的核心資料,用戶在系統中的日誌甚至是與其餘用戶的互動數據仍然可能被保留。
綜上所述,我們認為一個好的身份系統應該要能夠提供用戶積極授權的機制,並且解決用戶被遺忘權的問題。
\subsection{安全驗證}
人類最早就是使用密碼來進行身份驗證的,但是密碼的安全性一直是一個問題。論文\cite{6234436}指出基本上不存在安全的身份驗證技術,用戶勢必需要採用多種不同的驗證方式(MFA)來確保安全。然而,這樣的方式不僅增加了用戶的負擔,而且也降低了用戶的便捷性。我們認為一個好的身份系統應該要能夠提供一個安全的驗證機制,並且保證用戶的便捷性。
\subsection{平等自由}
論文\cite{preukschat2021self}提到了身份系統的平等自由問題,即身份系統的用戶與用戶之間,系統與系統之間甚至是用戶與系統之間擁有截然不同的立場。例如身份系統供應商期望可以獲取更多用戶的個資作為營運的獲利,而用戶則希望自己的個資不被濫用。一昧的偏向任何一方,只會導致系統環境的惡化或是對用戶的壟斷與壓迫。我們認為一個好的身份系統應該要能夠兼顧各類用戶與系統的立場,並且提供一個平等自由的環境。
\subsection{健全信任}
當前身份系統的困境之一在於難以確立健全的信任模型。人們往往用單一的憑證以二元的邏輯來判斷一個身份的真實性,然而真實世界的模糊\cite{s22155641}與雙向性\cite{4489846}使得這樣的模型難以適應。舉例來說如果同時存在五組憑證,其中存在 3 組驗證成功,2 組驗證失敗,那麼這個身份的真實性應該是多少呢?這樣的問題在當前的身份系統中是難以解決的。另外雙向性的問題常發生在去中心化系統中用戶發現的使用場景,即用戶必須要向系統證明自己的身份,但是系統也要向用戶證明自己的身份。我們認為一個好的身份系統應該要能夠提供一個信任模型,提供多維度且雙向的信任評估機制。
\section{近代身份驗證技術}
用戶體驗與安全驗證兼顧是近代資安驗證領域的發力重點之一。例如無密碼登入聯盟所提出的 FIDO2 協議就能在安全的前提下讓用戶輕易且快速地利用手邊裝置經由指紋或臉部辨識登入系統。或是基於風險的驗證(Risk-Based Authentication, RBA)技術,可以根據用戶的行為模式來判斷當前用戶行為的危險程度,進而提供不同的驗證方式。甚至是裝置指紋技術(Device printer)可以在所有呼叫時夾帶裝置本身的臨時數據,這將成為獨一無二的資訊,進而在每次的驗證中比對指紋差異程度來確保安全性。這些技術都能在不影響用戶體驗的前提下提供更高的安全性。
總結來說,近代身分驗證技術正朝著更安全、更便捷、更智能的方向發展。無密碼登入、基於風險的驗證和裝置指紋等技術的結合,為AID 應對當前複雜多變的網絡安全威脅提供了強有力的工具。
\section{當前身份系統的問題}
接著我們會介紹不同世代的身份系統設計,讓讀者更了解人們在用戶認知、平等自由、隱私保護和健全信任等方向上的努力。
\subsection{中心化身份}
典型的中心化身分系統包括早期的企業內部身分管理系統和政府的身分證系統。例如,傳統的 Windows Active Directory 就是一個典型的中心化身分管理系統。這樣的系統由一個中心化的機構控制,用戶只能通過這個機構來註冊與管理自己的身份。這樣的系統在用戶體驗上是非常簡單的,但是在其他方面還是存在著問題。例如用戶需要為了每個服務創建對應的帳戶身份,這將導致用戶認知的問題;又如用戶的身份被服務提供商完全控制,這導致用戶只能被動接受管理;再如用戶為了���服務被動創建新身份,不但供應商難以信任用戶,同時因為服務對用戶的不透明,用戶也難以信任服務本身。
\subsection{聯合身分}
聯合身分允許不同組織之間共享身分信息,使用戶能夠使用單一的憑證訪問多個系統。常見的聯合身分系統包括 SAML 和 WS-Federation。這樣的系統實現了單一登入(SSO)的機制,即用戶可以用同一組帳號進入無數的服務,大幅解決了用戶認知的問題,但是在其他方面還是存在著問題。例如用戶的身份被多個服務提供商共享,這意味著每個服務提供商都可以獲取用戶的所有個人信息,這無疑違背了 GDPR 中的最小權限原則。又如自由平等的問題,例如當不同服務共用同一套聯合身份系統,導致身份系統兼容不同服務的需求,一定對部分用戶或服務提供商造成不公平。
\subsection{使用者中心的身分}
常見的實作為 OpenID ,是一種開放標準和去中心化的驗證(Authentication)協議,代表了向用戶中心身份管理過渡的重要一步。OpenID允許用戶使用現有帳戶登錄多個服務,服務可以為了用戶創建對應的身份而不共享數據,只需要有對應的身份供應商提供驗證服務即可。
這解決了隱私保護的問題,因為確實由每個服務供應商分開來掌管用戶的數據,但是在其他方面還是存在著問題。例如所有身份都是以身份供應商的設置為準,依然無法兼容每個服務供應商的需求,甚至實務上大多數機構會為了每個新的用戶創建新的使用者中心身份,這又回過頭來增加了用戶的認知負擔。
\subsection{去中心化身分}
是最新的身份系統概念, 基於以太坊的 uPort 和微軟的 Microsoft ION 都是其實作。在用戶認知的部分透過 DID 的機制, 確保用同一個身份可以在不同的服務中使用, 在平等自由的部分也透過區塊鏈中統治代幣與可遷移性等機制, 保證了用戶與服務供應商之間的平等自由。甚至透過鏈上身份的評分與追蹤制度可以讓用戶與服務提供商互相信任,但是在其他方面還是存在著問題。例如用戶的身份被區塊鏈完全控制,這意味著用戶的身份無法被刪除,這違背了 GDPR 中的被遺忘權。又如用戶的所有身份驗證都是基於區塊鏈的,這意味著用戶只能使用各種加密錢包來進行網路活動,不管是忘記密碼還時私鑰被盜都將導致嚴重的後果,這些無疑使去中心化身份偏離使用者體驗的初衷。
\section{總結}
總而言之,我們認為當前的身份管理系統存在著許多問題,並且可以把問題分成六個面向如下:
\begin{enumerate}
  \item 用戶體驗
  \item 用戶認知
  \item 隱私保護
  \item 安全驗證
  \item 平等自由
  \item 健全信任
\end{enumerate}
其中任意方向的問題都存在著解法,但是解法往往會對其他方向產生影響。我們期望 AID 系統能夠在不影響其他方向的前提下解決所有問題,進而成為一個完美的身份系統。
