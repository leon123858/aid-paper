% !TeX root = ../main.tex

\chapter{文獻探討}
我們將討論當前身份系統面臨的核心困境,並提出引入"自主性"的概念作為可能的解決方向。最後,我們將提出一個身份技術的評估框架,用於客觀評估自主身份系統的優劣。
\section{身份系統的困境}
隨著網路技術的迅速發展,數位化身份管理已成為一個日益重要的議題。然而,當前網路系統面臨的核心問題在於其初始設計階段並未充分考慮身份管理的功能\cite{cameron2005laws}。這一事實導致了現有身份系統的諸多局限性,主要表現為通過軟體強制整合各種臨時性身份層作為身份驗證的基礎。這種做法不僅缺乏靈活性,更成為了身份技術發展過程中難以逾越的現實障礙。
理論上,一個理想的解決方案是存在的。類似於網路服務中廣泛採用的 TCP/IP 協議,我們期待能夠建立一個統一的標準來解決身份管理問題,使得開發者能夠基於此標準構建統一的身份系統。然而,這種期待在實踐中面臨著巨大挑戰。與網路傳輸不同,身份系統涉及的範疇更為廣泛和複雜。個人與組織之間存在著多樣化且往往相互衝突的需求,試圖通過單一標準來限制或規範這些需求是不切實際的。
綜上所述,我們認為當前的身份系統面臨著多重困境,這些困境不僅涉及技術層面,更與社會、文化和法律等諸多方面密切相關。為了應對這些挑戰,我們提出引入"自主性"(autonomy)的概念作為可能的解決方向。這一方法的核心在於:在滿足個人和組織多元需求的同時,逐步促進各方就身份管理達成共識。通過這種方式,我們有望以鬆耦合的架構構建出一個全新而統一的身份系統,既能適應當前的多樣化需求,又為未來的發展預留了足夠的靈活性。
\section{身份技術的評估}
為了分析本論文的自主身份系統是否確實解決了現有身份系統的問題,我們需要建立一個客觀的評估方法。遺憾的是,目前學術界和業界尚未就身份系統的評估達成共識,缺乏一個被廣泛認可的完整評量框架。這種情況凸顯了在身份系統研究領域中建立標準化評估方法的迫切需求。
鑒於此,本研究在綜合分析多個權威標準和指南的基礎上,提出了一個新的身份系統評估框架。這些參考標準包括但不限於:歐盟《通用數據保護條例》(GDPR)\cite{GDPR2016}、W3C的《去中心化身份標識符》(DID Core)規範\cite{DIDCore},以及美國國家標準與技術研究院(NIST)的《數字身份指南》\cite{NIST800-63-3}。
本評估框架圍繞六個核心維度展開,這些維度不僅反映了身份系統的技術特性,也涵蓋了社會和倫理層面的考量:
\begin{enumerate}
  \item 用戶體驗
  \item 用戶認知
  \item 隱私保護
  \item 安全驗證
  \item 平等自由
  \item 健全信任
\end{enumerate}
這六個維度被我們認為是構成優秀身份系統的基本特徵。這個多維度的評估框架旨在全面而客觀地評估本論文所設計的自主身份系統,並與現有系統進行比較分析。
在後續小節中,我們將詳細闡述每個評估維度的具體內容、評估標準以及在自主身份系統中的相關資訊。通過這種系統化的方法,我們期望能夠:提供一個全面的視角來審視自主身份系統的優勢與潛在限制
\subsection{用戶體驗}
用戶體驗在身份系統設計中扮演著關鍵角色,直接影響系統的可用性和採納率。Hamme等人\cite{inproceedings}的研究闡明了用戶體驗、安全性和隱私保護之間的複雜關係。該研究指出了一個普遍存在的現象:用戶傾向於選擇最簡單的方式來設置和使用身份系統,這種傾向可能導致系統安全性和隱私保護程度的降低。
然而,這種情況產生了一個兩難困境:
\begin{itemize}
  \item 為了提高安全性,強制用戶採用複雜的身份驗證方式可能會降低用戶體驗。
  \item 為了增強隱私保護,要求用戶完成詳細的隱私設置同樣可能導致用戶體驗的下降。
\end{itemize}
為了解決這一困境,研究者提出了"無摩擦驗證"(Zero Friction Authentication)的概念。這種新型驗證方法旨在:
\begin{enumerate}
  \item 最小化用戶在設置和使用過程中遇到的困難。
  \item 同時維持適當的安全性和隱私保護水平。
\end{enumerate}
無摩擦驗證的核心目標是在保護用戶安全和隱私的同時,顯著降低用戶的操作負擔。這種平衡對於現代身份系統的設計至關重要,因為它直接影響系統的使用率和效能。
為了解決這類問題,近代漸漸採用了許多新的技術,例如無密碼登入(FIDO2)、基於風險的驗證(RBA)、裝置指紋技術(Device printer)等,這些技術都能在不影響用戶體驗的前提下提供更高的安全性。我們將在後續研究中探討如何利用這些新技術在自主身份系統中實現無摩擦驗證,以及這種方法如何有效地平衡用戶體驗、安全性和隱私保護的需求。
\subsection{用戶認知}
用戶認知在數位身份管理中扮演著至關重要的角色,直接影響用戶對其線上身份的理解和管理能力。lastpass\cite{lastpass2020psychology}的研究揭示了用戶認知與實際情況之間存在顯著差距,這一差距可能導致嚴重的安全隱患和管理困難。
該研究的主要發現包括:
\begin{itemize}
  \item 用戶平均估計自己擁有 20 個線上帳號。
  \item 實際上,用戶平均擁有 37 個以上的帳號。
\end{itemize}
這種認知偏差帶來的後果是多方面的:
\begin{enumerate}
  \item \textbf{管理效率低下}:用戶難以有效管理超出其認知範圍的帳號,這讓用戶使用帳號的效率大打折扣。
  \item \textbf{安全風險增加}:用戶無法及時發現和處理超出管理範圍帳號的安全風險,這可能導致個人數據的外泄和濫用。
\end{enumerate}
儘管市場上存在各種密碼管理工具,但這些工具並未從根本上解決用戶認知的限制。這一現象突顯了設計更簡單、更直觀的身份管理系統的必要性。我們認為,未來的解決方案應該能夠:
\begin{itemize}
  \item 簡化多層次、多維度的用戶身份管理。
  \item 提高用戶對其數位身份的認知準確度。
  \item 在不增加認知負擔的情況下,增強安全性和便利性。
\end{itemize}
在後續研究中,我們將探討如何在自主身份系統的設計中解決這些認知挑戰,以及如何通過創新的方法來縮小用戶認知與實際情況之間的差距。
\subsection{隱私保護}
在數位時代,隱私保護已成為身份系統設計的核心考量之一。歐盟制定的《通用數據保護條例》(GDPR)代表了目前全球最嚴格的隱私保護標準。本研究認為,一個理想的身份系統應當能夠全面符合 GDPR 的要求,從而確保用戶隱私得到最大程度的保護。然而,近年來的 GDPR 違規案例表明,即便是大型企業也面臨著遵守某些 GDPR 規定的挑戰。
基於現有研究,我們發現以下兩個關鍵問題在當前身份系統中尤為突出:
\begin{enumerate}
  \item \textbf{用戶積極授權(Active Consent)}:Saemann\cite{saemann2022investigating}的研究強調,在當前的身份系統框架下,企業難以實現用戶對數據使用的明確授權。具體表現為:
        \begin{itemize}
          \item 企業難以證明其數據或權限使用行為被用戶充分授權。
          \item 用戶缺乏有效途徑證明自己的數據或權限被不當使用。
        \end{itemize}
  \item \textbf{被遺忘權(Right to be Forgotten)}:Smirnova\cite{smirnova2024understanding}指出,滿足用戶的被遺忘權要求面臨著巨大挑戰。主要原因包括:
        \begin{itemize}
          \item 用戶數據在系統中往往呈分散狀態。
          \item 即使刪除核心用戶的資料,仍可能保留用戶的系統日誌或與其他用戶的互動數據。
        \end{itemize}
\end{enumerate}
基於上述分析,我們提出符合現代隱私保護要求的身份系統應該具備以下特點:
\begin{itemize}
  \item 提供合理的機制讓用戶或企業能夠證明其數據使用行為符合或不符合授權。
  \item 提供有效的方法讓用戶行使被遺忘權,確保用戶不會被難以刪除的數據綁架。
\end{itemize}
在後續研究中,我們將詳細探討如何在自主身份系統中實現這些特性,以及如何在保護用戶隱私的同時,維持系統的可用性和效率。通過這種方式,我們旨在設計一個不僅符合 GDPR 要求,而且能夠適應未來隱私保護需求的先進身份系統。
\subsection{安全驗證}
身份驗證的安全性一直是身份系統設計中的核心挑戰。儘管密碼是最古老的驗證方式之一,但其固有的安全風險仍然存在。研究\cite{6234436}指出,實際上不存在完全安全的單一身份驗證技術。為了應對這一挑戰,多因素驗證(Multi-Factor Authentication, MFA)成為了廣泛採用的解決方案。然而,MFA 也帶來了新的問題:
\begin{itemize}
  \item 增加了用戶的操作負擔
  \item 降低了系統的整體便捷性
\end{itemize}
因此,我們認為理想的身份系統應該能夠在以下方面取得平衡:
\begin{enumerate}
  \item 提供高度安全的驗證機制
  \item 維持用戶友好的操作體驗
  \item 適應不同安全級別的靈活驗證策略
\end{enumerate}
在後續研究中,我們將探討如何在自主身份系統中實現這些特性,用靈活的策略滿足不同用戶在不同場景的需求,以此在安全的條件下提供更好的用戶體驗。
\subsection{平等自由}
身份系統中的平等自由問題是一個複雜的多方利益平衡問題。研究\cite{preukschat2021self}強調了身份系統中各方利益的衝突,主要表現在:
\begin{itemize}
  \item 用戶之間的權益差異
  \item 不同系統間的互操作性問題
  \item 用戶與系統供應商之間的利益衝突
\end{itemize}
例如,身份系統供應商可能希望獲取更多用戶個人資料以增加營收,而用戶則希望保護自己的隱私。這種利益衝突如果處理不當,可能導致:
\begin{itemize}
  \item 系統環境的惡化
  \item 對用戶權益的侵犯
  \item 市場壟斷和不公平競爭
\end{itemize}
因此,我們提出,一個理想的身份系統應該:
\begin{enumerate}
  \item 平衡各方利益,創造公平競爭環境
  \item 保護用戶權益,防止資料濫用
  \item 促進系統間的互操作性和資料可攜性
\end{enumerate}
在後續的研究中,我們將探討如何在自主身份系統中實現這些特性,以構建一個更公平、更開放的身份管理生態系統。
\subsection{健全信任}
在當前的身份系統中,建立健全的信任模型仍然是一個重大挑戰。傳統的二元邏輯驗證模式(即完全信任或完全不信任)已不能滿足現代身份系統的需求。這一問題主要體現在兩個方面:
\begin{enumerate}
  \item \textbf{真實世界的模糊性}:研究\cite{s22155641}指出,現實世界的身份驗證常常是模糊的,難以用簡單的二元邏輯來判斷。例如:
        \begin{itemize}
          \item 多組憑證同時存在,部分驗證成功,部分驗證失敗
          \item 不同憑證的可信度和重要性各不相同
        \end{itemize}
  \item \textbf{驗證的雙向性}:研究\cite{4489846}強調了身份驗證的雙向性問題,尤其在去中心化系統中表現突出:
        \begin{itemize}
          \item 用戶需要向系統證明自己的身份
          \item 系統同樣需要向用戶證明自己的合法性和可信度
        \end{itemize}
\end{enumerate}
基於以上分析,我們認為未來的身份系統應該具備以下特性:
\begin{itemize}
  \item 支持多維度、連續性的信任評估機制
  \item 實現雙向的身份驗證和信任建立過程
  \item 能夠適應不同場景下的信任需求,提供靈活的信任模型
\end{itemize}
在後續的研究中,我們將探討如何在自主身份系統中實現這些特性,以構建一個更加健全、更加可信的身份管理系統。
\section{總結}
綜上所述,當前的身份管理系統在不同方向都面臨著特定的挑戰。用戶體驗、用戶認知、隱私保護、安全驗證、平等自由和健全信任等方面的問題都需要得到解決。
值得注意的是,針對任一方向的解決方案往往會對其他方向產生連鎖影響。因此,設計一個能夠平衡所有這些方面的身份系統成為了一個極具挑戰性的任務。本研究提出的 AID(Autonomous Identity)系統旨在在不顯著影響其他方面的前提下,全面解決上述問題,以期實現一個更加完善的身份管理系統。
在接下來的章節中,我們將詳細介紹 AID 系統的設計理念、技術架構以及如何應對這六個核心面向的挑戰,以證明其作為下一代身份系統的潛力。
