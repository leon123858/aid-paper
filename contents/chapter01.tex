% !TeX root = ../main.tex

\chapter{緒論}
本研究探討了將自主概念應用於數位身份管理的創新方法,提出「自主身份」(Autonomous Identity,AID)系統。該系統使每位用戶在具備道德標準環境的身份系統中成為自己的唯一管理者,旨在解決當前身份管理模式的固有限制。
\section{研究背景}
自主(Autonomy)概念源於18世紀啟蒙運動,標誌著人類開始質疑對王權與神權的依賴,轉而追求通過科學與理性實現思想獨立與自由。啟蒙思想家伊曼努爾·康德(Immanuel Kant)對自主提出了開創性定義:「按照自己認同的道德標準自由行事」。這一定義至今仍廣泛被應用於政治、法律和教育等多個領域。接著,19世紀的重要哲學家弗里德里希·尼采(Friedrich Nietzsche)對道德提出了更深入的探討。他對道德的本質進行了批判性分析,認為道德源於弱者對強者的反抗,是一種透過集結群體的共識來約束個體行為的機制。

這兩位哲學家的思想不僅塑造了現代社會對自主的理解,也為我們重新思考數位時代的身份管理提供了重要的理論基礎。隨著網絡技術的飛速發展,個人身份資訊的管理與保護已成為當今社會的重要議題。在這一背景下,我們將自主的概念創新性地應用於數位身份管理領域。

本研究提出的「自主身份」(Autonomous Identity,AID)概念與目前主流探討的「自治身份」(Self-sovereign identity,SSI)系統有著本質的區別。SSI允許用戶參與身份管理系統的經營,賦予用戶對其數位身份的一定控制權。然而,AID則更進一步,使每位用戶在具備\textbf{道德標準環境}的身份系統中\textbf{成為自己的唯一管理者}。這種創新模式不僅賦予用戶更大的自主權,還在系統設計中納入了道德考量,以確保個人自主不會損害社會整體利益。
\section{研究目的與目標}
本研究旨在探討自主身份(Autonomous Identity,AID)作為解決當前身份管理問題的創新方案,並將「自主」的哲學思想融入數位身份管理的實踐中。具體而言,本研究將致力於實現以下目標:
\begin{enumerate}
  \item \textbf{分析現有數位身份系統的局限性}:深入探討當前身份管理系統的主要挑戰和固有限制。
  \item \textbf{構建AID系統的理論框架}:提出自主身份系統的核心理念,並闡述其如何解決現有系統的問題。
  \item \textbf{設計AID系統的技術架構}:提出一種能夠實現用戶完全自主管理的技術方案,包括去中心化存儲、智能合約等關鍵技術的應用。
  \item \textbf{評估AID系統的優勢與挑戰}:全面比較AID系統與傳統身份管理系統在多方面的差異,並分析AID系統在實際應用中可能面臨的挑戰。
  \item \textbf{提出AID系統的應用策略}:探討AID系統在不同領域(如金融、醫療、政務等)的潛在應用場景,並提出相應的實施策略和路線圖。
\end{enumerate}
最終,我們希望這項研究能夠推動數位身份管理領域的典範轉移,為構建更加安全、自由、便捷和公平的數位社會奠定基礎。
\section{論文架構}
為了基於自主的理念設計出一個完整的身分驗證系統,我們需要先了解現有的身分驗證技術,並且對於這些技術進行分析,找出其缺點(第二章)。接著提出我們的系統設計,並且說明其架構、資料結構與威脅模型(第三章)。最後,我們會透過實作來驗證我們的系統設計(第四章),最終我們會提出結論與對方案的未來展望(第五章)。