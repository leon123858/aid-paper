% !TeX root = ../main.tex

\chapter{緒論}
在當今數位時代,個人身分管理已成為一個日益重要且複雜的研究課題。隨著網絡技術的快速演進和社會對隱私保護的日益重視,傳統的身分管理系統面臨著前所未有的挑戰。本研究旨在探討一種特殊的解決方案——自主身分(Autonomous Identity,AID)系統,該系統融合了新的思想與先進技術,為數位身分管理帶來根本性的變革。通過重新審視「自主」的概念並將其應用於身分管理領域,本研究期望能夠為構建一個更自由、更公平的數位社會貢獻一份力量。
\section{研究背景}
自主(Autonomy)概念源於18世紀啟蒙運動,標誌著人類開始質疑對王權與神權的依賴,轉而追求通過科學與理性實現思想獨立與自由。啟蒙思想家伊曼努爾·康德(Immanuel Kant)對自主提出了開創性定義:「按照自己認同的道德標準自由行事」。這一定義至今仍廣泛被應用於政治、法律和教育等多個領域。接著,19世紀的重要哲學家弗里德里希·尼采(Friedrich Nietzsche)對道德提出了更深入的探討。他對道德的本質進行了批判性分析,認為道德源於弱者對強者的反抗,是一種透過集結群體的共識來約束個體行為的機制。

這兩位哲學家的思想不僅塑造了現代社會對自主的理解,也為重新思考數位時代的身分管理提供了重要的理論基礎。隨著網絡技術的迅速演進,個人身分資訊的管理與保護已成為當今社會的重要研究課題。在這一背景下,本研究將自主的概念顛覆性地應用於數位身分管理領域。

本研究提出的「自主身分」(Autonomous Identity,AID)概念與目前主流探討的「自治身分」(Self-sovereign identity,SSI)系統有著本質的區別。SSI允許使用者參與身分管理系統的經營,賦予使用者對其數位身分的一定控制權。然而,AID則更進一步,使每位使用者在具備\textbf{道德標準}的身分系統中\textbf{自由的管理自己}。這種模式不僅賦予使用者更大的自主權,還在系統設計中納入了道德考量,以確保個人自主不會損害社會整體利益。
\section{研究目的與目標}
本研究旨在探討自主身分(Autonomous Identity,AID)作為解決當前身分管理問題的候選方案,將「自主」的哲學思想融入數位身分管理的實踐中。具體而言,本研究將致力於實現以下目標:
\begin{enumerate}
  \item \textbf{分析現有數位身分系統的局限性}:深入探討當前身分管理系統的主要挑戰和固有限制。
  \item \textbf{構建AID系統的理論框架}:提出自主身分系統的核心理念,並闡述其如何解決現有系統的問題。
  \item \textbf{設計AID系統的技術架構}:提出一種能夠實現使用者完全自主管理的技術方案,包括去中心化存儲、智能合約等關鍵技術的應用。
  \item \textbf{評估AID系統的優勢與挑戰}:全面比較AID系統與傳統身分管理系統在多方面的差異,並系統性分析AID系統在實際應用中可能面臨的挑戰。
  \item \textbf{提出AID系統的未來展望}:探討AID系統在不同領域(如金融、醫療、政務等)的潛在應用場景,並提出相應的實施策略和路線圖。
\end{enumerate}
最終,本研究期望這項研究能夠推動數位身分管理領域的典範轉移,為構建更加安全、自由、便捷和公平的數位社會奠定基礎。
\section{主要貢獻}
與過去的研究不同,本研究引入「自主憑證」(Autonomous Certificate)\cite{NTU202102846}機制,完善了「自主身分」(Autonomous Identity,AID)系統\cite{ntu-lin2014autonomous},使AID系統在更大程度上解決了現有身分管理系統的問題。具體而言,本研究的主要貢獻包含以下幾個方面:
\begin{itemize}
  \item \textbf{提出「自主身分」(AID)系統架構:} 本研究提出了一個完整的三層架構,包含共識層、服務層和數據層,並詳細闡述了各層級的功能和設計理念
  \item \textbf{提出基於區塊鏈的去中心化評價機制:} 本研究提出了一套基於區塊鏈的去中心化評價機制,允許系統參與者互相評價,並將評價結果記錄在區塊鏈上,形成公開透明的信任體系
  \item \textbf{提出「數據層反轉」策略:} 本研究提出「數據層反轉」策略,將使用者資料的控制權從服務供應商交還給使用者,並透過「數據證書」機制實現資料在不同服務間的安全共享
  \item \textbf{開發AID系統概念驗證:} 本研究實現了AID系統的核心功能,並在受控的測試環境中進行了概念驗證,展示了AID系統在實際應用中的價值
\end{itemize}
\section{論文架構}
為了基於自主的理念設計出一個完整的身分驗證系統,本研究首先探討現有的身分驗證技術,並且對於這些技術進行分析,找出其缺點(第二章)。接著提出AID的系統設計,並且說明其架構細節與資料結構(第三章)。然後,透過實作來驗證本研究的系統設計(第四章),最終提出結論與對AID系統的未來展望(第五章)。