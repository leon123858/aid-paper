% !TeX root = ../main.tex

\chapter{緒論}
在當今數位時代,個人身分管理已成為一個日益重要且複雜的研究課題。隨著網絡技術的快速演進和社會對隱私保護的日益重視,傳統的身分管理系統面臨著前所未有的挑戰。本研究旨在完善自主身分(Autonomous Identity,AID)系統,該系統融合了新的思想與先進技術,為數位身分管理帶來根本性的變革。通過重新審視「自主」的概念並將其應用於身分管理領域,本研究期望能夠為構建一個更自由、更公平的數位社會貢獻一份力量。
\section{研究背景}
自主(Autonomy)概念源於18世紀啟蒙運動,標誌著人類開始質疑對王權與神權的依賴,轉而追求通過科學與理性實現思想獨立與自由。這一概念的演變不僅塑造了現代社會對自主的理解,也為重新思考數位時代的身分管理提供了重要的理論基礎。

% Therefore, persons have a special importance or worth , not because autonomy is an independently valuable or otherwise important property, but because the law commands that our maxims be such that we could will them to become universal law, which is the case if and only if they could be willed by any rational being as such.
Schaab等人\cite{Schaab2022}剖析了啟蒙思想家康德(Kant)對自主的理解:「因此,人具有特殊的重要性或價值,不僅僅是因為自主本身是一種獨立有價值或重要的特質,更因為道德法則要求我們的準則能夠成為普遍法則,而這只有在能被任何理性存在所意願時才能實現。」換言之,自主之所以重要,不僅在於其固有價值,更在於它能夠被所有理性存在所認同並成為普遍法則。這揭示了自主除了探討了個體的自由意志外,還關乎整個社會的道德基礎。

% Through resentment, the slaves were able to subvert moral authority away from the masters and label the masters as an evil group of people with no redeeming qualities. After the tra svaluation of values, the slaves have the upper hand in the moral sector. No longer were they an oppressed people, rather they have created new values: slave values. Such values were created through utilizing slave resentment for the masters. The slave's no saying to what the masters affirmed as good installed slave values, such as weakness, sickliness, and poverty, as being good values. Nietzsche says of the slave's victory, "Afterwards, when the rabble got the upper hand in Greece, fear ran rampant in their religion, too; and the ground was prepared for  7 Christianity" (BGE 49).
基於康德的觀點,可以更近一步探討何謂道德標準?這些標準如何建立?幸運的是,19世紀的重要哲學家尼采(Nietzsche)提出了「奴隸道德說」,為理解道德標準的形成提供了另一個視角。Haarer等人\cite{haarer2020nietzsche}分析道:「通過怨恨,奴隸們能夠顛覆主人的道德權威,將主人標籤為毫無可取之處的邪惡。在價值重估後,奴隸們在道德領域佔據了上風。他們不再是被壓迫的群體,而是創造了新的價值觀:奴隸價值觀。」

這揭示了道德標準的動態性和權力關係。在尼采看來,世界由多數弱者與少數強者組成,道德最初由強者制定,弱者被迫遵守。然而,弱者通過對強者進行道德譴責,重新定義了善惡標準,從而在道德領域獲得了主導權。這種轉變雖然在尼采看來是一種道德墮落,但也展示了弱者如何通過重新定義道德來獲得反抗強權的能力。

這些觀點挑戰了道德的絕對性與自由的全面性,並指出所謂對錯、善惡並非二元對立的絕對存在,而是建立在人類集體意識之上的相對概念。通過顛覆原有的階層秩序,新的平等與自由秩序得以建立。因此,本研究認為自主的真正含義可以理解為:「在理解並參與道德標準形成過程的基礎上,自由地遵循自己認同的道德準則行事。」

基於這些深刻的洞見,本研究提出的「自主身分」(Autonomous Identity,AID)概念與目前主流探討的「自治身分」(Self-sovereign identity,SSI)系統有著本質的區別。SSI允許使用者參與身分管理系統的經營,賦予使用者對其數位身分的一定控制權。然而,AID則更進一步,使每位使用者在具備\textbf{動態道德標準}的身分系統中\textbf{自由的管理自己}。這種模式不僅賦予使用者更大的自主權,還在系統設計中納入了道德考量,以確保個人自主不會損害社會整體利益。
\section{研究目的與目標}
本研究旨在探討自主身分(Autonomous Identity,AID)作為解決當前身分管理問題的候選方案,將「自主」的哲學思想融入數位身分管理的實踐中。具體而言,本研究將致力於實現以下目標:
\begin{enumerate}
  \item \textbf{分析現有數位身分系統的局限性}:深入探討當前身分管理系統的主要挑戰和固有限制。
  \item \textbf{構建AID系統的理論框架}:提出自主身分系統的核心理念,並闡述其如何解決現有系統的問題。
  \item \textbf{設計AID系統的技術架構}:提出一種能夠實現使用者完全自主管理的技術方案,包括去中心化存儲、智能合約等關鍵技術的應用。
  \item \textbf{評估AID系統的優勢與挑戰}:全面比較AID系統與傳統身分管理系統在多方面的差異,並系統性分析AID系統在實際應用中可能面臨的挑戰。
  \item \textbf{進行AID系統的概念驗證}:通過在區塊鏈上實作AID系統,驗證並模擬其在實際應用中的可行性和效果。
\end{enumerate}
最終,本研究期望這項研究能夠推動數位身分管理領域的典範轉移,為構建更加安全、自由、便捷和公平的數位社會奠定基礎。
\section{主要貢獻}
在本實驗室過去的研究中,學長Yuxuan\cite{ntu-lin2014autonomous}首先提出了「自主身分」(Autonomous Identity,AID)的概念。該概念透過將數據保留在客戶端的設計,實現了使用者對個人身份的自主權。然而,在他的設計中,使用者的身分仍需通過中心化機構進行驗證。為了解決這一問題,學長Tze-Nan\cite{NTU202102846}提出了「自主憑證」(Autonomous Certificate)機制,利用區塊鏈技術確保了身份驗證的去中心化,從而實現了更高程度的身份自主。

本研究結合前人的成果,進一步提出了AID系統的完整理論框架和系統架構,並通過實作驗證了系統在區塊鏈上的可行性。相較於以往的成果,本研究在系統設計和實作方面進行了深入的探討和優化,使AID系統能夠更全面地解決現有身分管理系統的問題。具體而言,本研究的主要貢獻包括以下幾個方面:
\begin{itemize}
  \item \textbf{提升使用者體驗 :} 透過深入分析AID系統的使用流程,優化了身份自主後產生的使用者體驗問題。
  \item \textbf{簡化使用者認知 :} 提出完整的AID與別名機制,使得使用者可以更加方便地管理自己的身份。
  \item \textbf{強化隱私保護 :} 通過融入「自主憑證」機制,確保被遺忘權、數據最小化等隱私原則的實現。
  \item \textbf{建立平等信任 :} 設計出基於區塊鏈的評價系統,使得多數弱勢使用者可以制衡少數強勢使用者。
\end{itemize}
\section{論文架構}
為了基於自主的理念設計出一個完整的身分驗證系統,本研究首先探討現有的身分驗證技術,並且對於這些技術進行分析,找出其缺點(第二章)。接著提出AID的系統設計,並且說明其架構細節與資料結構(第三章)。然後,透過實作來驗證本研究的系統設計(第四章),最終提出結論與對AID系統的未來展望(第五章)。