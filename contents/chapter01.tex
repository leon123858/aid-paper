% !TeX root = ../main.tex

\chapter{緒論}
在當今數位時代,個人身分管理已成為一個日益重要且複雜的研究課題。隨著網絡技術的快速演進和社會對隱私保護的日益重視,傳統的身分管理系統面臨著前所未有的挑戰。本研究旨在完善自主身分(Autonomous Identity,AID)系統,該系統融合了新的思想與先進技術,為數位身分管理帶來根本性的變革。通過重新審視「自主」的概念並將其應用於身分管理領域,本研究期望能夠為構建一個更自由、更公平的數位社會貢獻一份力量。
\section{研究動機}
自主(Autonomy)概念源於18世紀啟蒙運動,標誌著人類開始質疑對王權與神權的依賴,轉而追求通過科學與理性實現思想獨立與自由。這一概念的演變不僅塑造了現代社會對自主的理解,也為重新思考數位時代的身分管理提供了重要的理論基礎。

Campbell\cite{CAMPBELL2017381}剖析了啟蒙思想家康德(Kant)對自主的理解:「意志藉由自身成為自己的法則的那種特性」("that property of the will by which it [the will] is a law to itself")。換言之,自主可以被理解為「個體自由地遵循符合自己道德標準的法則」。這一解釋揭示了自主不僅涉及個體的自由意志,還關乎整個社會的道德基礎。

基於康德的觀點,可以衍伸出一個關鍵問題:如何在數位世界中形塑道德標準?這並非一個簡單的問題。本研究設想的做法是在身份系統中營造出一個孕育道德觀的空間,通過使用者之間的真實互動,逐漸形成共同認知的道德標準。這樣的系統將成為一個真正自主的身份系統,而非僅僅是當前普遍存在的資料庫式身份系統。

基於這些深刻的洞見,本研究提出的「自主身分」(Autonomous Identity,AID)概念與目前主流探討的「自治身分」(Self-sovereign identity,SSI)系統有著本質的區別。SSI允許使用者參與身分管理系統的經營,賦予使用者對其數位身分的一定控制權。然而,AID則更進一步,將「自主」的哲學思想融入數位身分管理的實踐中,使每位使用者在具備\textbf{動態道德標準}的身分系統中\textbf{自由的管理自己}。這種模式不僅賦予使用者更大的自主權,還在系統設計中納入了道德考量,以確保個人自主不會損害社會整體利益。
\section{主要貢獻}
本研究透過完善實驗室學長Yuxuan\cite{ntu-lin2014autonomous}提出的「自主身分」(Autonomous Identity,AID)系統。從自主認證、數據自主與信用評分三個方面出發,不僅建立更完整的理論基礎,更解決了過去並未關注到的重要議題。此外,通過在區塊鏈OurChain\cite{ourlab408_ourchain}上實作AID系統,驗證了其在實際應用中的可行性和效果。本研究的主要貢獻包括:
\begin{itemize}
  \item \textbf{構建更完整的AID理論基礎}:在Yuxuan的研究基礎上,進一步完善了AID系統的理論框架,涵蓋自主認證、數據自主和信用評分等多個方面。
  \item \textbf{設計更具體的AID系統架構}:提出了一種符合最新發展趨勢的AID系統架構,成功解決了資訊安全、隱私保護、使用者體驗等多方面的技術難題。
  \item \textbf{實作基於區塊鏈的AID系統原型}:在區塊鏈上實作了AID系統的原型,並通過實際應用驗證了其可行性和效果。
\end{itemize}
\section{論文架構}
為了基於自主的理念設計出一個完整的身分驗證系統,本研究首先探討現有的身分驗證技術,並且對於這些技術進行分析,找出其缺點(第二章)。接著提出AID的系統設計,並且說明其架構細節與資料結構(第三章)。然後,透過實作來驗證本研究的系統設計(第四章),最終提出結論與對AID系統的未來展望(第五章)。