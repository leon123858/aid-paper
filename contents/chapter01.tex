% !TeX root = ../main.tex

\chapter{緒論}
本研究探討了將自主概念應用於數位身份管理的創新方法,提出「自主身份」(Autonomous Identity,AID)系統。該系統賦予每位用戶全面控制個人數位身份的能力,旨在解決當前身份管理模式的固有限制。
\section{研究背景}
自主(Autonomy)概念源於18世紀啟蒙運動,標誌著人類開始質疑對王權與神權的依賴,轉而追求通過科學與理性實現思想獨立與自由。啟蒙思想家伊曼努爾·康德(Immanuel Kant)對自主提出了開創性定義:「按照自己的法則行事」。這一定義至今仍廣泛應用於政治、法律和教育等多個領域。

康德的自主概念強調個體應基於自身道德觀制定規則,並主動依循這些規則與他人互動。本研究旨在將這一富有哲學內涵的概念融入當代數位身份系統設計,以解決現有系統中的諸多困境。

本研究提出的「自主身份」(Autonomous Identity,AID)概念與目前主流探討的「自治身份」(Self-sovereign identity,SSI)系統有著本質的區別。SSI允許用戶參與身份管理過程,而AID則更進一步,使每位用戶\textbf{成為自己數位身份的唯一管理者}。這種創新方法確保用戶即使在複雜的虛擬網絡環境中,也能完全掌控自己的數位身份,不會因技術限制而喪失自主權。本文將深入探討自主身份系統的理論基礎、具體實施方法,以及它對現有數位身份管理體系可能帶來的革命性影響。
\section{研究目的}
本研究旨在探討自主身份(Autonomous Identity,AID)作為解決當前身份管理問題的潛在方案。具體而言,本文將:
\begin{enumerate}
  \item 分析當前數位身份系統的主要挑戰和限制。
  \item 評估自主身份系統對比傳統身份管理系統的優勢。
  \item 提出自主身份系統的核心概念和技術基礎。
  \item 討論自主身份系統在實際應用中可能面臨的挑戰和解決方案。
\end{enumerate}
\section{研究意義}
自主身份系統代表了身份系統領域的一個重要典範轉移。在過去的身份管理系統中,每個用戶被由上而下的方式賦予一個身份。而自主身份系統則轉向由用戶自己控制自己的身份,由下而上地組織成群,形成一個又一個的身份互動節點,最終透過去中心化的方式使每個節點針對想要共享的資訊取得共識。

通過賦予用戶更多對自身數位身份的自主權,自主身份系統有潛力改變個人、組織和社會之間的互動方式。本研究的結果將為政策制定者、技術開發者和最終用戶提供寶貴的見解,有助於推動更進步的身份管理創新。
\section{論文架構}
為了基於自主的理念設計出一個完整的身分驗證系統,我們需要先了解現有的身分驗證技術,並且對於這些技術進行分析,找出其缺點(第二章)。接著提出我們的系統設計,並且說明其架構、資料結構與威脅模型(第三章)。最後,我們會透過實作來驗證我們的系統設計(第四章)。