% !TeX root = ../main.tex

\chapter{緒論}
本研究探討了將自主概念應用於數位身份管理的創新方法,提出「自主身份」(Autonomous Identity, AID)系統,通過為用戶提供對其數位身份的全面控制權,以解決當前身份管理模式的固有限制。
\section{研究背景}
自主(Autonomy)的概念源於18世紀啟蒙時代,當時人類開始質疑對王權與神權的依賴,轉而追求通過科學與理性達成思想上的獨立與自由。啟蒙思想家伊曼努爾·康德(Immanuel Kant)對自主提出了開創性的定義,即「按照自己的法則來行事」。這一定義至今仍廣泛應用於政治、法律、教育等諸多領域。

康德的自主概念強調個體應基於自身的道德觀制定規則,並主動依循這些規則與他人互動。本研究旨在將這一深具哲學意涵的概念引入當代數位身份系統的設計中,以解決現有系統中的諸多困境。

我們提出「自主身份」(Autonomous Identity, AID)的概念,作為實踐康德自主理念的一種方法。自主身份系統旨在確保每位用戶能夠「成為自己的主人」,即使在虛擬的網絡環境中,也不會因數位身份的特性而喪失自主權。本文將探討自主身份系統的理論基礎、實施方法及其對現有數位身份管理體系的潛在影響。
\section{研究目的}
本研究旨在探討自主身份(Autonomous Identity, AID)作為解決當前身份管理問題的潛在方案。具體而言,本文將:
\begin{enumerate}
  \item 分析當前數位身份系統的主要挑戰和限制。
  \item 評估自主身份系統對比傳統身份管理系統的優勢。
  \item 提出自主身份系統的核心概念和技術基礎。
  \item 討論自主身份系統在實際應用中可能面臨的挑戰和解決方案。
\end{enumerate}
\section{研究意義}
自主身份系統代表了身份系統領域的一個重要典範轉移。過去的身份管理系統主要由機構控制,每個用戶被由上而下的方式賦予一個身份。而自主身份系統則轉向由用戶自己控制自己的身份,由下而上的組織成群,形成一個又一個的身份互動節點,最終透過去中心化的方式使每個節點針對想要共享的資訊取得共識。

通過賦予用戶更多對自身數位身份的自主權,自主身份系統有潛力改變個人、組織和社會之間的互動方式。本研究的結果將為政策制定者、技術開發者和最終用戶提供寶貴的見解,有助於推動更進步的身份管理創新。
\section{論文架構}
為了要基於自主的理念設計出一個完整的身分驗證系統,我們需要先了解現有的身分驗證技術,並且對於這些技術進行分析,找出其缺點(chapter2)。接著提出我們的系統設計,並且說明其架構、資料結構與威脅模型(chapter3)。最後,我們會透過實作來驗證我們的系統設計(chapter4)。\newpage