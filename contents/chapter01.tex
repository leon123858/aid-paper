% !TeX root = ../main.tex

\chapter{緒論}

自主(Autonomy)的核心概念可以簡單地定義為「成為自己的主人」。在身份系統中,我們致力於實現這一目標。其中,自主身份(Autonomous Identity, AID)即為「成為自己身份的主人」,這一理念的實作。

\section{研究背景}

在數位時代,身份管理已成為一個日益重要且複雜的議題。隨著在線服務的普及,個人數據的價值和風險都在不斷增加。傳統的身份驗證方法,如用帳戶名和密碼,已經顯得不夠安全和便捷。同時,中心化的身份管理系統也面臨著壟斷、隱私侵犯等問題。

近年來,區塊鏈技術的發展為去中心化身份管理提供了新的可能性。自主身份系統(AID)正是在這一背景下應運而生。AID 旨在將身份的控制權交還給個人,使用戶能夠完全掌控自己的數位身份和個人數據。這一概念不僅有助於增強隱私保護和數據安全,還能夠簡化身份驗證流程,提高用戶體驗。

然而,實現真正的自主身份系統仍面臨諸多挑戰,包括技術實現、法律法規、用戶習慣等方面的問題。因此,深入研究自主身份系統的可行性和潛在影響具有重要的理論和實踐意義。

\section{研究目的}

本研究旨在探討自主身份(Autonomous Identity, AID)作為解決當前身份管理問題的潛在方案。具體而言,本文將:

\begin{enumerate}
  \item 分析當前數位身份系統的主要挑戰和限制。
  \item 探討自主身份系統的核心概念和技術基礎。
  \item 評估自主身份系統在便捷性、安全性、公平性與隱私性方面的優勢。
  \item 討論自主身份系統在實際應用中可能面臨的挑戰和解決方案。
\end{enumerate}

\section{研究意義}

自主身份系統代表了身份系統領域的一個重要範式轉移。過去的身份管理系統主要由機構控制,每個用戶被由上而下的方式賦予一個身份。而自主身份系統則轉向由用戶自己控制自己的身份,由下而上的組織成群,形成一個又一個的身份互動節點,最終透過去中心化的方式連結一個又一個身份節點而形成更大的身份網絡。

通過賦予用戶更多對自身數位身份的自主權,AID 有潛力改變個人、組織和社會之間的互動方式。本研究的結果將為政策制定者、技術開發者和最終用戶提供寶貴的見解,有助於推動更進步的身份管理創新。

\section{論文架構}

為了要基於自主的理念設計出一個完整的身分驗證系統,我們需要先了解現有的身分驗證技術,並且對於這些技術進行分析,找出其缺點(chapter2)。接著提出我們的系統設計,並且說明其架構、資料結構、威脅模型以及商業流程(chapter3)。最後,我們會透過實作來驗證我們的系統設計(chapter4)。\newpage