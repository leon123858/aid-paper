% !TeX root = ../main.tex

\begin{abstract}
  當前的數位身分管理體系正面臨前所未有的危機。大型科技公司和政府機構對個人數據的壟斷控制,不僅嚴重侵蝕了用戶的數位自由,更導致了普遍的隱私缺失問題。用戶被迫在獲取服務與保護個人信息間做出艱難抉擇,而他們的數位足跡不斷被收集、分析和利用,往往超出其知情和同意的範圍。這種失衡的權力結構不僅威脅個人權益,更可能導致社會的信任崩塌和動搖。
  
  本研究直面這些緊迫挑戰,提出了一種革命性的解決方案——自主身分(AID)系統。AID系統旨在從根本上重塑數位身分管理的權力結構,將控制權從壟斷實體轉移到個體用戶手中。通過創新的去中心化架構和先進的區塊鏈技術,AID系統為用戶提供了前所未有的數據自主權。更重要的是,AID系統引入了基於共識的動態道德框架,旨在約束和平衡各方行為。這一道德體系不僅保護用戶權益,還通過建立公平的遊戲規則來抑制強者的過度行為,從而創造更加公平和負責任的數位生態系統。
  
  初步的概念驗證表明,AID系統有潛力徹底改變用戶與數位服務的互動方式,不僅實現真正的數位自由和隱私保護,還能促進更加公正和負責任的數位社會形成。然而,作為一項顛覆性技術,AID系統的廣泛採用仍面臨來自既得利益者的阻力和社會認知等挑戰。儘管如此,本研究為解決數位時代最緊迫的身分管理和道德問題開闢了新方向,不僅對學術界具有重要意義,更為構建一個更加公平、自由、安全和道德的數位社會提供了可能性。
\end{abstract}

\begin{abstract*}

  Abstract Abstract Abstract Abstract Abstract Abstract Abstract Abstract Abstract Abstract Abstract Abstract Abstract Abstract Abstract Abstract Abstract Abstract Abstract Abstract Abstract Abstract Abstract Abstract Abstract Abstract Abstract Abstract Abstract Abstract Abstract Abstract Abstract Abstract Abstract Abstract Abstract Abstract Abstract Abstract Abstract Abstract Abstract Abstract Abstract Abstract Abstract Abstract Abstract Abstract Abstract Abstract Abstract Abstract Abstract Abstract Abstract Abstract Abstract Abstract Abstract Abstract Abstract

\end{abstract*}