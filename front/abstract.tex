% !TeX root = ../main.tex

\begin{abstract}
  現代數位身分系統面臨嚴峻挑戰:身分驗證漏洞威脅用戶安全,中央化數據儲存易遭攻擊導致大規模個資外洩,大型組織壟斷關鍵服務造成權力失衡。這些問題不僅危及個人權益,更阻礙了數位社會的發展。本研究將完善「自主身分」系統,旨在徹底重塑數位身分管理。本研究從身分認證、資料管理和信用評分三個關鍵領域著手,設計了一套去中心化解決方案,成功將數位身分的控制權從大型機構手中歸還給個人用戶,顯著提升了用戶自主權。本研究還基於區塊鏈OurChain進行了概念驗證,成功證實了AID系統的可行性。本研究認為「自主身分」系統有潛力徹底改變人們與數位世界的互動方式,為建立一個更安全、公平和自由的數位社會鋪平道路。
\end{abstract}

\begin{abstract*}
  The modern identity systems have confronted with serious challenges: authentication vulnerabilities threaten user security, centralized data storage is vulnerable to large-scale data breaches, and core services monopolized by large organizations bring about power imbalance. These issues not only endanger individual rights but also hinder the development of digital society. This research aims to improve the "Autonomous Identity" system, with the goal of fundamentally reshaping digital identity management. Focusing on three crucial areas: identity authentication, data management, and credit scoring. We design a decentralized solution that returns the control of digital identity from large institutions to individual users, significantly enhancing user autonomy. We also conducted a proof of concept based on the OurChain, successfully demonstrating the feasibility of the AID system. This study believes that the "Autonomous Identity" system has the potential to completely change the way people interact with the digital world, paving the way for a safer, fairer, and freer digital society.
\end{abstract*}