% !TeX root = ../main.tex

\begin{abstract}
  身分管理已成為一個日益重要且複雜的議題。傳統的身份系統面臨諸多挑戰,包含身份驗證過程依賴外部服務,威脅使用者的安全與隱私。或是中心化的數據儲存容易成為攻擊目標,導致大規模的個資外洩。還有大型服務的壟斷使得使用者在面對擁有權力的大型機構時,難以保護自身權益等。這些問題不僅威脅到個人,更限制了整個社會的發展。針對這些挑戰,本研究完善了「自主身分」(Autonomous Identity,AID)系統,從自主認證、數據自主與信用評分三個方面著手,致力於重塑數位身分管理的權力結構,將主權從壟斷者手中歸還給使用者。AID是由使用者個人裝置管理的網路唯一身份,賦予使用者充分的自主權。我們通過在區塊鏈OurChain上實作AID系統,不僅建立了更完整的理論基礎,更驗證了其在實際應用中的可行性和效果。期待這項研究能為數位時代的身分管理帶來革新,促進個人隱私權的保護,並推動更自由、公平的數位社會發展。
\end{abstract}

\begin{abstract*}
  Identity management has become an increasingly important and complex issue. Traditional identity systems face numerous challenges, including authentication processes that rely on external services, threatening users' security and privacy. Centralized data storage is vulnerable to attacks, leading to large-scale personal information breaches. Moreover, the monopoly of large-scale services makes it difficult for users to protect their rights when confronted with powerful large institutions. These problems not only threaten individuals but also limit the development of society as a whole. To address these challenges, this research has refined the "Autonomous Identity" (AID) system, focusing on three aspects: autonomous authentication, data autonomy, and credit scoring. It aims to reshape the power structure of digital identity management, returning autonomy from monopolists to users. AID is a unique network identity managed by users' personal devices, granting users full autonomy over their identity. By implementing the AID system on the OurChain blockchain, we have not only established a more complete theoretical foundation but also verified its feasibility and effectiveness in practical applications. We hope this research will bring innovation to identity management in the digital age, promote the protection of personal privacy rights, and foster the development of a freer and fairer digital society.
\end{abstract*}