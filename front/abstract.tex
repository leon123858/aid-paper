% !TeX root = ../main.tex

\begin{abstract}
  在當今數位時代,個人身分管理已成為一個日益重要且複雜的議題。數位世界正面臨嚴峻的身分管理危機,科技巨擘和政府機構掌握了龐大的個人資料,引發了嚴重的隱私憂慮。在這樣的環境下,人們經常陷入接受服務和維護個人隱私的兩難困境。更令人不安的是,人們的網路行為經常在未經充分知情同意的情況下被記錄、分析,甚至遭到不當利用。這種不平等的權力分配不僅威脅到個人權益,更動搖了整個社會賴以維繫的道德基礎。

  本研究直面這些挑戰,引入「自主憑證」(Autonomous Certificate)機制,完善了「自主身分」(Autonomous Identity,AID)系統,從根本上重塑數位身分管理的權力結構,將主權從壟斷者手中歸還給個體使用者。AID系統致力於讓每位使用者在具備道德標準的身分系統中自由地管理自身,賦予使用者對其身分相關數據的完全控制權。

  AID系統作為一項顛覆性技術,其廣泛採用仍面臨諸多挑戰,包括來自既得利益者的阻力和社會認知方面的障礙。儘管如此,本研究仍為解決數位時代最緊迫的身分管理問題開闢了新的方向。
\end{abstract}

\begin{abstract*}
  In today's digital age, personal identity management has become an increasingly important and complex issue. The digital world is facing a identity management crisis, with tech giants and government agencies controlling vast amounts of personal data, raising serious privacy concerns. People often find themselves in a dilemma between accepting services and maintaining personal privacy. More disturbingly, people's online behaviors are frequently recorded, analyzed, and even misused without their full informed license. This unequal distribution of power not only threatens individual rights, but also damages the moral foundation on which society depends.

  This research solves these challenges by introducing the 'Autonomous Certificate' mechanism, which perfects the 'Autonomous Identity' (AID) system, fundamentally reshaping the power structure of digital identity management and returning sovereignty from monopolists to individual users. The AID system is committed to allowing each user to freely manage their identity within an kind identity system, granting users complete control over their identity-related data.

  As a breakthrough technology, the widespread adoption of the AID system still faces many challenges, including resistance from vested interests and barriers in social cognition. Nevertheless, this research opens up new directions for addressing the most pressing identity management issues in the digital age.
\end{abstract*}