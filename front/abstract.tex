% !TeX root = ../main.tex

\begin{abstract}
  現今的數位世界正面臨一場身分管理危機,科技巨頭和政府機構掌控了大量個人資料,帶來了深重的隱私憂慮。在這樣的環境下,人們常常陷入兩難:要麼放棄某些服務,要麼犧牲個人隱私。更令人不安的是,我們的網路行為總是被暗中記錄、剖析,甚至被加以利用,而這些行為往往超出了我們原本同意的範圍。這種不平等的權力分配,不只威脅到個人的權益,更動搖了整個社會賴以維繫的道德基礎。

  本研究直面這些挑戰,提出了一種革命性的解決方案——自主身分(AID)系統。AID系統希望從根本上重塑數位身分管理的權力結構,將主權從壟斷者轉移到個體使用者手中。通過創新的機制和先進的區塊鏈技術,AID系統不僅為使用者提供了身分的自由,還引入了基於共識的動態道德準則,通過建立公平的遊戲規則來抑制強者的壟斷行為。

  AID系統作為一項顛覆性技術,其廣泛採用仍面臨諸多挑戰。這些挑戰包括來自既得利益者的阻力,以及社會認知方面的障礙。儘管如此,本研究為解決數位時代最緊迫的身分管理問題開闢了新的方向。
\end{abstract}

\begin{abstract*}

  Abstract Abstract Abstract Abstract Abstract Abstract Abstract Abstract Abstract Abstract Abstract Abstract Abstract Abstract Abstract Abstract Abstract Abstract Abstract Abstract Abstract Abstract Abstract Abstract Abstract Abstract Abstract Abstract Abstract Abstract Abstract Abstract Abstract Abstract Abstract Abstract Abstract Abstract Abstract Abstract Abstract Abstract Abstract Abstract Abstract Abstract Abstract Abstract Abstract Abstract Abstract Abstract Abstract Abstract Abstract Abstract Abstract Abstract Abstract Abstract Abstract Abstract Abstract

\end{abstract*}