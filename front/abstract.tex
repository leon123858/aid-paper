% !TeX root = ../main.tex

\begin{abstract}
  現代數位身分系統面臨嚴峻挑戰:身分驗證漏洞威脅用戶安全,中央化數據儲存易遭攻擊導致大規模個資外洩,大型組織壟斷關鍵服務造成權力失衡。這些問題不僅危及個人權益,更阻礙了數位社會的發展。本研究將完善「自主身分」系統,旨在徹底重塑數位身分管理。本研究從身分認證、資料管理和信用評分三個關鍵領域著手,設計了一套去中心化解決方案,成功將數位身分的控制權從大型機構手中歸還給個人用戶,顯著提升了用戶自主權。本研究還基於區塊鏈OurChain進行了概念驗證,成功證實了AID系統的可行性。本研究認為「自主身分」系統有潛力徹底改變人們與數位世界的互動方式,為建立一個更安全、公平和自由的數位社會鋪平道路。
\end{abstract}

\begin{abstract*}
  Today's digital identity systems have big problems: weak security puts users at risk, central data storage can be easily hacked, and big companies have too much control. These issues hurt people and slow down digital progress. Our research improves the 'Autonomous Identity' (AID) system to fix these problems. We created a new solution that focuses on three main areas: checking who you are, managing your data, and scoring your reputation. This system gives control back to users, not big companies. We tested AID using OurChain blockchain and showed it works. We consider AID can change how people use the digital world, helping create a safer, fairer, and freer online society.
\end{abstract*}