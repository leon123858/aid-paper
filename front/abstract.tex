% !TeX root = ../main.tex

\begin{abstract}
  當前的數位身分管理系統正面臨危機。大型科技公司和政府機構對個人數據的壟斷,不僅侵蝕了使用者的數位自由,更導致了嚴重的隱私問題。使用者被迫在獲取服務與保護個人數據間做出抉擇,而他們的數位足跡不斷被收集、分析和利用,往往超出同意的範圍。這種失衡的權力結構不僅威脅個人自由,更動搖了整個社會的道德基礎。
  
  本研究直面這些挑戰,提出了一種革命性的解決方案——自主身分(AID)系統。AID系統希望從根本上重塑數位身分管理的權力結構,將主權從壟斷者轉移到個體使用者手中。通過創新的架構和先進的區塊鏈技術,AID系統不僅為使用者提供了身份的自由,還引入了基於共識的動態道德準則,通過建立公平的遊戲規則來抑制強者的壟斷行為。

  AID系統作為一項顛覆性技術,其廣泛採用仍面臨諸多挑戰,包括來自既得利益者的阻力以及社會認知方面的障礙。儘管如此,本研究為解決數位時代最緊迫的身份管理問題開闢了新的方向。
\end{abstract}

\begin{abstract*}

  Abstract Abstract Abstract Abstract Abstract Abstract Abstract Abstract Abstract Abstract Abstract Abstract Abstract Abstract Abstract Abstract Abstract Abstract Abstract Abstract Abstract Abstract Abstract Abstract Abstract Abstract Abstract Abstract Abstract Abstract Abstract Abstract Abstract Abstract Abstract Abstract Abstract Abstract Abstract Abstract Abstract Abstract Abstract Abstract Abstract Abstract Abstract Abstract Abstract Abstract Abstract Abstract Abstract Abstract Abstract Abstract Abstract Abstract Abstract Abstract Abstract Abstract Abstract

\end{abstract*}