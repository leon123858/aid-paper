% !TeX root = ../main.tex

\begin{abstract}
  在當今數位時代,身分管理已成為一個日益重要且複雜的議題。科技巨擘和政府機構掌握了龐大的個人資料,引發了嚴重的隱私安全問題。更令人不安的是,人們的網路行為經常在未經充分知情同意的情況下被記錄、分析,甚至遭到不當利用。這種不平等的現狀不僅威脅到個人權益,更動搖了整個社會賴以維繫的道德基礎。本研究直面這些挑戰,引入「自主憑證」(Autonomous Certificate)機制,完善了「自主身分」(Autonomous Identity,AID)系統,從根本上重塑數位身分管理的權力結構,將主權從壟斷者手中歸還給使用者。AID系統致力於讓每位使用者在具備道德標準的身分系統中自由地管理自身,賦予使用者對其身分相關數據的完全控制權。期待能為數位時代的身分管理帶來革新,促進個人隱私權的保護,並推動更自由、公平的數位社會發展。
\end{abstract}

\begin{abstract*}
  In today's digital age, personal identity management has become an increasingly important and complex issue. Tech giants and government agencies control vast amounts of personal data, and raise serious privacy concerns. More disturbingly, people's online behaviors are frequently recorded, analyzed, and even misused without their full informed license. This unequal status quo not only threatens individual rights, but also damages the moral foundation on which society depends.This research solves these challenges by introducing the 'Autonomous Certificate' mechanism, which perfects the 'Autonomous Identity' (AID) system, fundamentally reshaping the power structure of digital identity management and returning sovereignty from monopolists to individual users. The AID system is committed to allowing each user to freely manage their identity within an kind identity system, granting users complete control over their identity-related data. It is expected to bring innovation to digital identity management in the digital age, promote the protection of individual privacy, and promote the development of a more free and fair digital society.
\end{abstract*}