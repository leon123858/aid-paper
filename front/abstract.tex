% !TeX root = ../main.tex

\begin{abstract}
  現代數位身分系統面臨嚴峻挑戰:身分驗證漏洞威脅使用者安全,中央化數據儲存易遭攻擊導致大規模個資外洩,大型組織壟斷關鍵服務造成權力失衡。這些問題不僅危及個人權益,更阻礙了數位社會的發展。本研究將完善「自主身分」的系統設計,旨在徹底重塑數位身分管理。本研究從身分認證、資料管理和信用評分三個領域著手,設計了一套去中心化解決方案,成功將數位身分的控制權從大型機構手中歸還給個人使用者,顯著提升了使用者自主權。本研究還基於區塊鏈OurChain進行了概念驗證,成功證實了AID系統的可行性。本研究認為「自主身分」系統有潛力徹底改變人們與數位世界的互動方式,為建立一個更安全、公平和自由的數位社會鋪平道路。
\end{abstract}

\begin{abstract*}
  The modern identity systems face significant challenges: authentication vulnerabilities threaten user security, centralized data storage is susceptible to large-scale breaches, and key services monopolized by large organizations create power imbalances. These issues not only jeopardize individual rights but also impede the development of a digital society. This research aims to enhance the system design of "Autonomous Identity" to fundamentally reshape digital identity management. It focuses on three areas: identity authentication, data management, and credit scoring. We propose a decentralized solution that transfers control of digital identity from large institutions to individual users, significantly enhancing user autonomy. A proof of concept based on OurChain successfully demonstrates the feasibility of the Autonomous Identity system. This study posits that the Autonomous Identity system has the potential to transform digital interactions, paving the way for a safer, fairer, and freer digital society.
\end{abstract*}
致謝 3
摘要 5
Abstract 7
目次 9
圖次 15
表次 17
第一章 緒論 1
1.1 研究動機 . . . . . . . . . . . . . . . . . . . . . . . . . . . . . . . . . 2
1.2 主要貢獻 . . . . . . . . . . . . . . . . . . . . . . . . . . . . . . . . . 3
1.3 論文架構 . . . . . . . . . . . . . . . . . . . . . . . . . . . . . . . . . 3
第二章 文獻探討 5
2.1 身分系統的起源 . . . . . . . . . . . . . . . . . . . . . . . . . . . . . 5
2.2 身分系統的迭代 . . . . . . . . . . . . . . . . . . . . . . . . . . . . . 7
2.2.1 中心化身分 . . . . . . . . . . . . . . . . . . . . . . . . . . . . . . 7
2.2.2 聯合身分 . . . . . . . . . . . . . . . . . . . . . . . . . . . . . . . 8
2.2.3 使用者中心的身分 . . . . . . . . . . . . . . . . . . . . . . . . . . 9
2.2.4 自治身分 . . . . . . . . . . . . . . . . . . . . . . . . . . . . . . . 10
2.2.5 未來展望 . . . . . . . . . . . . . . . . . . . . . . . . . . . . . . . 11
2.3 AID 系統的發展 . . . . . . . . . . . . . . . . . . . . . . . . . . . . . 11
2.3.1 最初的自主身分 . . . . . . . . . . . . . . . . . . . . . . . . . . . 12
2.3.2 自主憑證機制 . . . . . . . . . . . . . . . . . . . . . . . . . . . . 13
2.4 身分系統的挑戰 . . . . . . . . . . . . . . . . . . . . . . . . . . . . . 14
2.4.1 使用者體驗 . . . . . . . . . . . . . . . . . . . . . . . . . . . . . . 14
2.4.2 使用者認知 . . . . . . . . . . . . . . . . . . . . . . . . . . . . . . 16
2.4.3 隱私保護 . . . . . . . . . . . . . . . . . . . . . . . . . . . . . . . 17
2.4.4 平等信任 . . . . . . . . . . . . . . . . . . . . . . . . . . . . . . . 18
2.4.5 法律合規性 . . . . . . . . . . . . . . . . . . . . . . . . . . . . . . 19
2.4.6 公認原則 . . . . . . . . . . . . . . . . . . . . . . . . . . . . . . . 20
2.5 本章總結 . . . . . . . . . . . . . . . . . . . . . . . . . . . . . . . . . 21
第三章 系統設計 23
3.1 系統的新設計 . . . . . . . . . . . . . . . . . . . . . . . . . . . . . . 23
3.1.1 自主認證 . . . . . . . . . . . . . . . . . . . . . . . . . . . . . . . 24
3.1.1.1 最簡自主認證 . . . . . . . . . . . . . . . . . . . . . . 24
3.1.1.2 MFA 的加入 . . . . . . . . . . . . . . . . . . . . . . . 26
3.1.1.3 AID Server 的加入 . . . . . . . . . . . . . . . . . . . 28
3.1.1.4 自主憑證的使用 . . . . . . . . . . . . . . . . . . . . 29
3.1.1.5 自主認證流程 . . . . . . . . . . . . . . . . . . . . . . 31
3.1.2 數據自主 . . . . . . . . . . . . . . . . . . . . . . . . . . . . . . . 32
3.1.2.1 數據被遺忘權 . . . . . . . . . . . . . . . . . . . . . . 33
3.1.2.2 數據明確授權 . . . . . . . . . . . . . . . . . . . . . . 36
3.1.2.3 數據可驗證性 . . . . . . . . . . . . . . . . . . . . . . 38
3.1.2.4 無特權的執行 . . . . . . . . . . . . . . . . . . . . . . 41
3.1.3 信用評分 . . . . . . . . . . . . . . . . . . . . . . . . . . . . . . . 42
3.1.3.1 信用評分機制 . . . . . . . . . . . . . . . . . . . . . . 43
3.1.3.2 生態系的營運 . . . . . . . . . . . . . . . . . . . . . . 43
3.2 系統架構設計 . . . . . . . . . . . . . . . . . . . . . . . . . . . . . . 44
3.2.1 系統結構概覽 . . . . . . . . . . . . . . . . . . . . . . . . . . . . 44
3.2.2 層次與角色對應 . . . . . . . . . . . . . . . . . . . . . . . . . . . 45
3.2.2.1 共識層與共識核心 . . . . . . . . . . . . . . . . . . . 45
3.2.2.2 服務層與服務提供者 . . . . . . . . . . . . . . . . . . 46
3.2.2.3 數據層與終端使用者 . . . . . . . . . . . . . . . . . . 46
3.2.3 共識層 . . . . . . . . . . . . . . . . . . . . . . . . . . . . . . . . 46
3.2.4 服務層 . . . . . . . . . . . . . . . . . . . . . . . . . . . . . . . . 47
3.2.4.1 身分管理 . . . . . . . . . . . . . . . . . . . . . . . . 47
3.2.4.2 憑證管理 . . . . . . . . . . . . . . . . . . . . . . . . 49
3.2.4.3 數據管理 . . . . . . . . . . . . . . . . . . . . . . . . 49
3.2.5 數據層 . . . . . . . . . . . . . . . . . . . . . . . . . . . . . . . . 50
3.2.5.1 身分管理 . . . . . . . . . . . . . . . . . . . . . . . . 50
3.2.5.2 數據管理 . . . . . . . . . . . . . . . . . . . . . . . . 51
3.2.5.3 憑證管理 . . . . . . . . . . . . . . . . . . . . . . . . 51
3.2.5.4 數據存儲 . . . . . . . . . . . . . . . . . . . . . . . . 52
3.3 系統設計細節 . . . . . . . . . . . . . . . . . . . . . . . . . . . . . . 53
3.3.1 區塊鏈憑證機制 . . . . . . . . . . . . . . . . . . . . . . . . . . . 53
3.3.1.1 自主憑證 . . . . . . . . . . . . . . . . . . . . . . . . 53
3.3.1.2 數據憑證 . . . . . . . . . . . . . . . . . . . . . . . . 55
3.3.2 無摩擦機制 . . . . . . . . . . . . . . . . . . . . . . . . . . . . . . 56
3.3.2.1 身分識別問題 . . . . . . . . . . . . . . . . . . . . . . 56
3.3.2.2 基於使用者時空的分析方法 . . . . . . . . . . . . . . 57
3.3.2.3 基於危險程度的驗證機制 . . . . . . . . . . . . . . . 58
3.3.2.4 混合數據管理 . . . . . . . . . . . . . . . . . . . . . . 59
3.3.3 密碼救援問題 . . . . . . . . . . . . . . . . . . . . . . . . . . . . 60
3.3.3.1 極限多因素驗證 . . . . . . . . . . . . . . . . . . . . 60
3.3.4 組織使用者控管 . . . . . . . . . . . . . . . . . . . . . . . . . . . 61
3.4 資料結構 . . . . . . . . . . . . . . . . . . . . . . . . . . . . . . . . . 62
3.4.1 共識核心 . . . . . . . . . . . . . . . . . . . . . . . . . . . . . . . 62
3.4.2 AID Server . . . . . . . . . . . . . . . . . . . . . . . . . . . . . . 62
3.4.3 Wallet . . . . . . . . . . . . . . . . . . . . . . . . . . . . . . . . . 64
3.5 本章總結 . . . . . . . . . . . . . . . . . . . . . . . . . . . . . . . . . 64
第四章 系統實作 65
4.1 系統架構 . . . . . . . . . . . . . . . . . . . . . . . . . . . . . . . . . 65
4.2 實現細節 . . . . . . . . . . . . . . . . . . . . . . . . . . . . . . . . . 66
4.3 流程分析 . . . . . . . . . . . . . . . . . . . . . . . . . . . . . . . . . 68
4.3.1 產生新的 AID 與自主憑證 . . . . . . . . . . . . . . . . . . . . . 68
4.3.2 進入支付服務獲取收據 . . . . . . . . . . . . . . . . . . . . . . . 69
4.3.3 使用 AI 服務對話 . . . . . . . . . . . . . . . . . . . . . . . . . . 70
4.4 本章總結 . . . . . . . . . . . . . . . . . . . . . . . . . . . . . . . . . 71
第五章 結論 73
參考文獻 75
附錄 A — 實際操作介面 83
A.1 AID 錢包 . . . . . . . . . . . . . . . . . . . . . . . . . . . . . . . . . 83
A.2 AI 聊天軟體 . . . . . . . . . . . . . . . . . . . . . . . . . . . . . . . 86
附錄 B — 系統 UML 圖 91
B.1 物件圖 . . . . . . . . . . . . . . . . . . . . . . . . . . . . . . . . . . 91
B.2 時序圖 . . . . . . . . . . . . . . . . . . . . . . . . . . . . . . . . . . 91
B.3 流程圖 . . . . . . . . . . . . . . . . . . . . . . . . . . . . . . . . . . 98