% !TeX root = ./main.tex

% --------------------------------------------------
% 資訊設定(Information Configs)
% --------------------------------------------------

\ntusetup{
	university*   = {National Taiwan University},
	university    = {國立臺灣大學},
	college       = {電資工程學院},
	college*      = {College of Engineering},
	institute     = {資訊工程所},
	institute*    = {Department of Computer Science and Information Technology},
	title         = {基於OurChain的自主身分系統設計與實作},
	title*        = {Design and Implementation of Autonomous Identity System Based on OurChain},
	author        = {林俊佑},
	author*       = {Jun You, Lin},
	ID            = {R11922114},
	advisor       = {薛智文},
	advisor*      = {Chih-Wen (Steven) Hsueh},
	% date          = {2020-05-01},         % 若註解掉,則預設為當天
	oral-date     = {2024-07-25},         % 若註解掉,則預設為當天
	%   DOI           = {10.5566/NTU2018XXXXX},
	keywords      = {自主身分, 管理, 認證, 隱私, 區塊鏈},
	keywords*     = {Autonomous Identity, Management, Authentication, Privacy, Blockchain},
}

% --------------------------------------------------
% 加載套件(Include Packages)
% --------------------------------------------------

\usepackage[sort&compress]{natbib}      % 參考文獻
\usepackage{amsmath, amsthm, amssymb}   % 數學環境
\usepackage{ulem}                       % 下劃線、雙下劃線與波浪紋效果
\usepackage{booktabs}                   % 改善表格設置
\usepackage{multirow}                   % 合併儲存格
\usepackage{diagbox}                    % 插入表格反斜線
\usepackage{array}                      % 調整表格高度
\usepackage{longtable}                  % 支援跨頁長表格
\usepackage{paralist}                   % 列表環境
\usepackage{graphicx}                   % 圖

\usepackage{lipsum}                     % 英文亂字
\usepackage{zhlipsum}                   % 中文亂字

\usepackage{algorithm}									% 演算法
\usepackage{algpseudocode}							% 程式碼
\usepackage{enumitem}										% 列表環境

% --------------------------------------------------
% 套件設定(Packages Settings)
% --------------------------------------------------

% 全局设置 enumerate 环境
\setlist[enumerate,itemize]{itemsep=0pt, topsep=0pt, parsep=0pt}
